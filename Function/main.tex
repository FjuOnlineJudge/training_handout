\section{函式與遞迴}
函式為程式裡的運算單元,可以接受資料,並回傳指定值。main是C/C++程式的入口函式,接受命令列的參數,正常情況會回傳0代表正常運作。\\
以下為其語法
\lstinputlisting{Function/function0.cpp}
範例
\lstinputlisting{Function/function1.cpp}
函式有個特性為自呼叫,也就是自己的區域可以呼叫自己,但要有終止條件,不然會陷入無限遞迴,同時也要避免遞迴過深,造成stack overflow。
\lstinputlisting{Function/function2.cpp}

\subsection{參考}
參考型態代表一個變數的別名,可直接取得變數的位址,並間接透過參考型態別名來操作物件, 作用類似於指標,但卻不必使用指標語法,也就是不必使`*`運算子來提取值。
\lstinputlisting{Function/reference.cpp}
參考型態可用在取代太長的變數(如:`a[x][y][z]`),容易維護。另一個是當函式要傳入可修改的值,可取代指標。
\subsection{傳值}
函式傳入的參數,可以是一般、指標或是參考型態,以下以Swap來介紹
\subsubsection{call by value}
傳入的變數為一般型態,會"複製"一份到函式,原本的變數不會有任何改變。
\lstinputlisting{Function/pass1.cpp}
\subsubsection{call by address/value of pointer}
傳入的變數為指標型態,函式內的變數改變,是對記憶體操作,所以原本的數字也會跟著改變。
\lstinputlisting{Function/pass2.cpp}
\subsubsection{call by reference}
傳入的變數為參考型態,函數內的變數是原本變數的分身,所以函數內變數改變時,原本變數也會跟者改變。
\lstinputlisting{Function/pass3.cpp}

函式有很多用處,一個為模組化,意即相同的部分(最多只差一些參數),寫成一個函式,除了簡潔,在除錯也比較方便。一個是利用自呼叫特性實作遞迴,遞迴可將問題拆解成同類的子問題而解決問題。\\
常見遞迴使用
\begin{enumerate}
\item 分治
\item dp中的top-down
\item 圖/樹的搜索
\end{enumerate}
