\section{struct}
struct 可將原本獨立的資料包在一起。例如:三維空間由x座標、y座標、z座標組成。\\
語法:
\begin{enumerate}
\item 型態(type)可以是一般或是指標型態
\item 也可以寫函式或重載運算子
\end{enumerate}
\lstinputlisting{Struct/struct1.cpp}
以下的例子為平面上的點。
\lstinputlisting{Struct/struct2.cpp}
\subsection{建構子(constructor)、解構子(destructor)}
建構子和strcut name同名,是用來初始化struct裡的資料,如果不寫的話,會有預設建構子,裡面的資料都是亂數。根據請況可多載,然而,如果你寫了運算子,一定要寫一個不帶任何參數的運算子,否則的話,像第14行這樣只有宣告,沒加其他東西的的程式碼就不會通過。\\
解構子的名字形式為 $ \sim strcut name$,是在變數離開作用域時運作,不寫的話也是會有預設解構子,在程式比賽中這樣就已足夠。
\subsection{重載運算子}
c++原有的型態都根據需要,定義了各種運算子,但 struct 如果有需要的話,須自己定義。而在競賽中,常需要作排序而需要小於運算子(`<`)。