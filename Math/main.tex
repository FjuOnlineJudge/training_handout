\section{數論}
\subsection{質數}
質數問題在程式競賽中常常遇到,通會建立質數表來查詢質數。
\subsubsection{一般篩法}
每找到一個數x,就知道2x, 3x, 4x...都不是質數,把他們從候選名單剃除。
\lstinputlisting{Math/PrimeTable1.cpp}
複雜度可到$O(N\log\log N)$,如果不從平方開始剃除,則會退化至$O(N\log N)$
\subsubsection{線性篩法}
將一般篩法優化至$O(N)$,我們期望每個合數都被其最小質因數剔除,這樣可以確保其均攤的線性。實作上,我們讓每一個數去提除"自己乘上$<=$其質因數的所有質數"的數字即可。
\lstinputlisting{Math/PrimeTable2.cpp}
\subsection{因數}
\subsubsection{因數個數}

\subsubsection{因數總和}

\subsubsection{最大公因數和最小公倍數}

\subsection{歐拉函數}

\subsection{模}
\subsubsection{性質}

\subsubsection{模逆元}

\subsubsection{中國剩餘定理}
\subsection{快速冪}

\subsection{一些關於質數的定理}
\subsubsection{費馬小定理}
\subsubsection{歐拉定理}
\subsubsection{Wilson's theorem}