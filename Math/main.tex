\section{數論}
數論是數學中,專門在研究整數性質的領域,涉及質數、組合等,被譽為最純的數學領域,是數學的皇后。
\subsection{質數}
質數為因數只有1和自己的數,質數問題在程式競賽中常常遇到,通會建立質數表來查詢質數。
\subsubsection{一般篩法}
每找到一個質數x,就知道2x, 3x, 4x...都不是質數,把他們從候選名單剃除。
\lstinputlisting{Math/PrimeTable1.cpp}
複雜度可到$O(N\log\log N)$,如果不從平方開始剃除,則會退化至$O(N\log N)$
\subsubsection{線性篩法}
將一般篩法優化至$O(N)$,我們期望每個合數都只被其最小質因數剔除,這樣可以確保其均攤的線性。實作上,我們讓每一個數去替除"自己乘上$<=$其質因數的所有質數"的數字即可。
\lstinputlisting{Math/PrimeTable2.cpp}
\subsection{因數}
我們能將任意一個正整數做質因數分解,形式為$N=P_{1}^{x_{1}}P_{2}^{x_{2}}P_{3}^{x_{3}}...=\Pi P_{i}^{x_{i}}$,根據此形式,我們可以求出任一正整數的因數個數及因數總和
\begin{enumerate}
\item [因數個數] $(x_{1}+1)(x_{2}+1)(x_{3}+1)...=\Pi (x_{i}+1)$
\item [因數總和] $(P_{1}^{0}+P_{1}^{1}+...+P_{1}^{x_{1}})(P_{2}^{0}+P_{2}^{1}+...+P_{2}^{x_{2}})...=\Pi\Sigma_{j=0}^{x_{i}}(P_{i}^{j})$
\end{enumerate}
\subsection{最大公因數和最小公倍數}
最大公因數(Greatest Common Divisor, GCD),可以用輾轉相除法求得。
\lstinputlisting{Math/GCD.cpp}
複雜度為$O(\log(min(a,b)))$,最慘狀況發生在兩數為費式數列相鄰項時,C++<algorithm>有內建的\_\_gcd可以用。\\
最小公倍數(Least Common Multiple,LCM),則為兩數相乘再除以他們的GCD,為避免溢位狀況,可先將一數除以GCD,再乘以另外一數。
\subsubsection{題目}
\begin{enumerate}
\item zerojudge a007(質因數分解)
\item TIOJ 1040 (連分數)
\end{enumerate}
\subsection{模}
\subsubsection{性質}
\begin{enumerate}
\item [加法] $(a+b)\ \%\ p= (a\ \%\ p +b\ \%\ p )\ \%\ p$
\item [減法] $(a-b)\ \%\ p= (a\ \%\ p -b\ \%\ p +p)\ \%\ p$
\item [乘法] $(a*b)\ \%\ p= (a\ \%\ p *b\ \%\ p )\ \%\ p$
\item [次方] $(a^b)\ \%\ p= ((a\ \%\ p )^b)\ \%\ p$
\item [加法結合律] $((a+b)\ \%\ p+c)\ \%\ p = (a+(b+c))\ \%\ p$
\item [乘法結合律] $((a*b)\ \%\ p*c)\ \%\ p = (a*(b*c))\ \%\ p$
\item [加法交換律] $(a+b)\ \%\ p=(b+a)\ \%\ p$
\item [乘法交換律] $(a*b)\ \%\ p=(b*a)\ \%\ p$
\item [結合律] $((a+b)\ \%\ p*c)= ((a*c)\ \%\ p +(b*c)\ \%\ p )\ \%\ p$
\end{enumerate}
如果$a\equiv b (\mod m)$,我們會說$a,b$在模$m$下同餘,關於同餘也有一些性質。\\
參考文章:https://wangwilly.github.io/willywangkaa/2018/05/08/Discrete-mathematics-Chinese-Remainder-Theorem/
\begin{enumerate}
\item [整除性] $a\equiv b \quad (\mod m) \Rightarrow c \cdot m  = a - b , c \in \mathbb{Z}$\\ $\Rightarrow a \equiv b\quad ( \mod m ) \Rightarrow m \; | \; a-b$
\item [遞移性] 若 $a \equiv b \quad (\mod c) , b \equiv d \quad (\mod c)$\\ 則 $a \equiv d (\mod c)$
\item [保持基本運算] $\left \{ \begin{matrix} a \equiv b (\mod m)\\ c \equiv d (\mod m)\end{matrix}\right. \Rightarrow \left\{\begin{matrix}a \pm c \equiv b \pm d (\mod m)\\ a \cdot c \equiv b \cdot d (\mod m)\end{matrix}\right.$
\item [放大縮小模數] 令$k \in \mathbb{Z}^+ , a \equiv b \quad (\mod m) \Leftrightarrow k \cdot a \equiv k \cdot b \quad (\mod k \cdot m)$
\end{enumerate}
\subsubsection{快速冪}
我們常常遇到求$a^b mod\ c$為多少的題目,最簡單的作法是用迴圈花b次算出答案,但是在b很大時就無法快速算出。這時如果拆成$a^1,a^2,a^4,...,a^{2^x}$,先分別計算在乘起來,這樣只要花費$O(\log b)$的時間即可。
\lstinputlisting{Math/abc.cpp}
\subsubsection{模逆元}
模逆元是取模下的反元素,即為找到$a^{-1}$使得$aa^{-1}\equiv\ 1\ mod\ c$。如果a要在mod c下有反元素,那麼$gcd(a,c)=1$,根據貝祖定理,可知存在整數$x,y$,使得$ax+cy=gcd(a,c)=1$,這裡的$x$即為a的反元素,我們可以修改找GCD的辦法,找出$a$的反元素。
\lstinputlisting{Math/extgcd.cpp}
\subsubsection{題目}
\begin{enumerate}
\item TIOJ 1199
\item zerojudge b430
\item zerojudge a289(求模逆元)
\end{enumerate}
\subsection{一些定理}
\subsubsection{歐拉函數}
歐拉函數計算對於一個整數N,小於等於N的正整數中,有幾個和N互質。通常用$\Phi(n)$表示。其公式為$\Phi(N)=N\times\Pi_{p|N}(1-\frac{1}{p})$\\
\lstinputlisting{Math/Phi.cpp}
歐拉函數還有一些性質\\
如果 p 是質數:$\Phi(p) = p-1$以及$\Phi(p^k)=p^{k−1}\times (p−1)$\\
如果 pq 互質:$\Phi(pq) = \Phi(p)\times\Phi(q)$\\
根據這些性質,我們能寫出一個類似質數篩法的作法。
\lstinputlisting{Math/PhiTable.cpp}
\subsubsection{費馬小定理}
給定一個質數$p$及一個整數$a$,那麼:\\
$a^p \equiv a (\mod p)$\\
如果$gcd(a,p)=1$,則:\\
$a^{p-1} \equiv 1 (\mod p)$\\
\subsubsection{歐拉定理}
歐拉定理是比較generate版本的費馬小定理。給定兩個整數n和a,如果$gcd(a,n)=1$,則\\
$a^{\Phi(n)} \equiv 1 (\mod n)$\\
如果n是質數,$\Phi(n)=n-1$,也就是費馬小定理。
\subsubsection{Wilson's theorem}
給定一個質數$p$,則:\\
$(p-1)!\equiv -1 (\mod p)$
\subsubsection{題目}
\begin{enumerate}
\item UVa 12493(歐拉定理)
\item UVa 1434(Wilson's theorem)
\end{enumerate}
\subsection{中國剩餘定理}
中國剩餘定理,又稱中國餘數定理,是數論中的一個關於一元線性同餘方程組的定理。用來解決像下面這種問題:\\
"有物不知其數,三三數之剩二,五五數之剩三,七七數之剩二。問物幾何?",這題答案為$23+105n,n>=0$\\
列出這種問題的式子(設$m_i$兩兩互質):
$\left \{ \begin{matrix} X\equiv a_1 (mod\ m_1)\\ X\equiv a_2 (mod\ m_2)\\ \cdot \\ \cdot \\ X\equiv a_n (mod\ m_n)\end{matrix}\right.$\\
解決這類問題最簡單適用枚舉來求解,不過如果範圍太大就會吃TLE了。因此我們 ,我們設$M=\Pi_{i=1}^{n} m_i$,令$x_1\ mod\ m_1\equiv a_1$,而對其他$m_i$皆整除,拆解$x_1$成為$x_1=(M/m_1)r_1$亦即$(M/m_1)r_1\equiv a_1 (mod\ m_1)$,用找模逆元的方式求出$x_1$。$x_2,x_3,...,x_n$也是依照這個辦法求出,最後$X=(\Sigma_{i=1}^{n} x_i)\ mod\ M\ +\ kM\ (k>=0)$。\\
總共有n個式子,每個式子都要找一次模逆元,所以複雜度為$O(n \log m)$
\subsubsection{題目}
\begin{enumerate}
\item zerojudge c641
\end{enumerate}