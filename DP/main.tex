\section{動態規劃}
動態規劃(Dynamic Programming, DP)和遞迴一樣,會把問題分解成子問題,不同的是DP同樣的子問題會重複,這時會以"時間換取空間"的方式,把答案存起來,之後用到時不用再次進行計算。
\subsection{特性}
\begin{enumerate}
\item 重複子問題:相同的一個子問題,需要多次查詢。
\item 最佳子結構:當問題被拆成若干個規模較小的問題時,可以透過這些子問題得到這個問題的最佳解。
\item 無後效性:子問題不會互相呼叫,一定存在一個能完整求值的計算順序。
\end{enumerate}
\subsection{步驟}
DP通常會被定義成一個算出答案的函數,函數的參數個數可變,對於一種題目可能有不同種的定義辦法,要看自己如何決定。
\begin{enumerate}
\item 訂出狀態
\item 導出轉移:從一個子問題的答案推到另一個
\item 設好基底:有些問題明顯有答案(也許無解),就直接算出答案
\end{enumerate}
而時間複雜度,通常就會是「使用到的狀態個數」乘上「轉移時間」了。
\subsection{費式數列}
費式數列是個看到膩的數列,我們就把他的DP式列出來
\begin{enumerate}
\item 狀態:$f(x)$代表費式數列第$n$項
\item 轉移:$f(n)=f(n-1)+f(n-2)$
\item 基底:$f(n)=n,where$ $n\leq 1$
\end{enumerate}
我們先用遞迴的方式寫寫看
\lstinputlisting{DP/fib01.cpp}
實際測試,會發現數字到40左右就跑不太動,原因是因為他有很多重複的子問題,那麼我們就以"空間換時間"的辦法來加速。
\lstinputlisting{DP/fib02.cpp}
這種用遞迴實作的叫做Top-down,好處是容易寫出來,缺點是遞迴的常數非常大。\\
而費式數列還有一種用迴圈的寫法。
\lstinputlisting{DP/fib03.cpp}
這種的迴圈實作的叫做Buttom-up,雖然比較難寫,但是時間常數比遞迴小,有時可以利用"滾動陣列"來壓縮空間。\\
第一個單純遞迴的版本,其複雜度等同費式數列第N項。第二、三版本,每項費式數列只會被算一次,所以複雜度為$O(N)$。
\subsection{滾動陣列}
如果一個DP式只會用到臨近的項數,就可以捨棄其他用不到的,藉此壓縮空間。
例如:費式數列只會用到前兩項,我們就用取餘數來實作滾動陣列,注意被餘數要保持正數。
\lstinputlisting{DP/fib04.cpp}
\subsection{題目}
\begin{enumerate}
\item UVa10003(區間DP)
\item UVa10285
\item zerojudge d212
\item UVa11310
\end{enumerate}
接下來我們談論一些經典問體
\subsection{背包問題}
背包問題中,會給你背包的容量C,以及N樣物品,它們有不同的重量w和價值v,背包裡物品價值總和最大為何?
\subsubsection{0/1背包}
01背包限制每樣物品最多只能放一個。那就來寫DP式。
\begin{enumerate}
\item 狀態:$dp[n][i]$ $=$ 使用 $1 ∼ n$ 個物品湊出重量 i 時,所可得到的最大價值。
\item 轉移:$dp[n][i] = max(dp[n − 1][i − w[n]] + v[n], dp[n − 1][i])$
\item 基底:$dp[0][0] = 0$, $dp[0][i] = −INF$ when $i>0$
\end{enumerate}
以防在DP的過程不小心用到,湊不出來的狀態要設成負無限大,或是使用if判斷取代預設。
\lstinputlisting{DP/01backpack1.cpp}
現在時間空間複雜度皆為$O(NC)$,而空間可以在持續優化。\\
記得剛剛說過的滾動陣列嗎?先回去看看DP式,$dp[n]$轉移只會用到$dp[n-1]$,所以我們只要保留前一行(行列的區分我不管)就好了。
\lstinputlisting{DP/01backpack2.cpp}
這樣的空間複雜度為$2C$,再來,我們發現第二個維度,他只會用到小於它的地方,那麼我們可以只用一行。但是要由後往前轉移,這樣才不會讓還沒轉移到的狀態,取到已轉移的,保證該項物品只會被取一次。
\lstinputlisting{DP/01backpack3.cpp}
\subsubsection{無限背包}
無限背包每樣物品能放無限多個。同樣要先列出DP式。
\begin{enumerate}
\item 狀態:$dp[n][i]$ $=$ 使用 $1 ∼ n$ 個物品湊出重量 i 時,所可得到的最大價值。
\item 轉移:$dp[n][i] = max(dp[n − 1][i − w[n]] + v[n], dp[n][i − w[n]] + v[n], dp[n − 1][i])$
\item 基底:$dp[0][0] = 0, dp[0][i] = −INF$ when $i>0$
\end{enumerate}
如果第i個物品重量為$w_i$,那麼最多可以放$\frac{C}{w_i}$次,要多一層迴圈來模擬。
\lstinputlisting{DP/infbackpack1.cpp}
時間複雜度為$O(NC\Sigma\frac{C}{w_i})$\\
我們現在回到之前01背包最後一個範例,當初要由後往前,是為了保證每個物品都只放了一個,而現在我們要放越多越好,那我們就由前面開始轉移就行了。
\lstinputlisting{DP/infbackpack2.cpp}
時間複雜度為$O(NC)$
\subsubsection{有限背包}
有限背包每樣物品能放限制最多可以放$a_i$個。這裡請讀者自己列出DP式。
我們一樣由01背包延伸,一樣物品最多可以放$a_i$個,那麼迴圈最多要跑$a_i$次,這樣整體時間複雜度為$O(NC\Sigma a_i)$。\\
有限背包又沒有像無限背包那麼好的性質,但如果我們帶入二進位的觀念,第$i$位代表$2^{i-1}$,我們把$a_i$個物品拆成$1,2,4...,2^k$次方個枚舉,那麼我們可以將時間複雜度降為$O(NC\Sigma(\log a_i))$
講道更進階一點的DP,會學到"單調隊列優化",能將時間複雜度降為$O(NC)$
\subsubsection{題目}
\begin{enumerate}
\item zerojudge b184(01背包)
\item UVa10664(01背包)
\item UVa10898(無限背包)
\item Uva10086(有限背包)
\item TIOJ 1387(有限背包)
\end{enumerate}
\subsection{最長共同子序列 (Longest Common Subsequence)}
子序列是指一個序列去除某些元素後所形成的新序列(當然也可以不刪除任何東西),而所謂的最長共同子序列,則是要求兩個字串去除元素後變成相同的序列,最長可以為何。這題觀念是簡單的,在這裡就列出dp式供讀者自行解讀。
\begin{enumerate}
\item 狀態:$dp[i][j]$ $=$ 使用 $a[1]~a[i]$和$b[1]~b[j]$ 的LCS長度。
\item 轉移:$dp[i][j] =\left\{ \begin{array}{cc}
dp[i-1][j-1]+1 & if\ a[i]=b[j]\\
max(dp[i-1][j],dp[i][j-1]) & else
\end{array} \right\}$
\item 基底:$dp[i][0]=dp[0][i]=0$ when $i \geq 0$
\end{enumerate}
\subsubsection{題目}
\begin{enumerate}
\item Uva10405
\end{enumerate}
\subsection{最長遞增子序列 (Longest Increasing Subsequence)}
給你一個序列長度為N,求最長遞增子序列的長度。\\
你可能會在其他地方看見一維DP式,在這裡不提,有興趣可以去演算法筆記看,這裡要介紹另一種DP式
\begin{enumerate}
\item 狀態:$dp[n][i]$ $=$ 使用 $1 ∼ n$ 個數字湊出長度 $i$ 的 LIS,末端數字最小為何。
\item 轉移:$dp[n][i] =\left\{ \begin{array}{cc}
min(dp[n-1][i],a[n]) & if\ a[n]>dp[n-1][i-1]\\
dp[n-1][i] & else
\end{array} \right\}$
\item 基底:$dp[0][0] = -INF, dp[0][i] = INF, dp[i][0]=don't$ $care$ when $i \geq 1$
\end{enumerate}
時間複雜度為$O(n^2)$。\\
再來,我們令$g[i]=dp[x][i]$, where $1\leq x \leq N$,我們會發現$g[i]$為一個嚴格遞減序列,而且改的數字剛好的$a[i]$,更妙的是,被改的數字為$lower_bound(a[i])$,我們就可以用二分搜來得到$O(n\log n)$的演算法了。
\lstinputlisting{DP/LIS.cpp}
\subsubsection{題目}
\begin{enumerate}
\item UVa11456
\item UVa11790
\end{enumerate}