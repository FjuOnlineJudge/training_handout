\section{演算法}
\subsection{何謂演算法}
簡言之就是解決問題的方法,用程式語言把他明確地列出。
\subsection{枚舉}
枚舉是最直觀的演算法,將有可能的答案都搜過一遍,當然沒有頭緒的搜尋可能會得到龐大的複雜度,要根據題目的性質來降低複雜度。
\subsubsection{回朔}
枚舉有時能用遞迴實作,在遇到不可能的情形馬上回傳,這種方法就叫做回朔ㄊ。
\subsubsection{特殊枚舉方式}
\begin{enumerate}
\item [二進位] 利用二進位來表示集合內有哪些元素要用,進而枚舉所有元素子集,但受限於時間複雜度$O(2^n)$,集合的元素個數通常只有30個(甚至15個)。   
\item [字典序] 利用 next\_permutation 或 prev\_permutation 達到枚舉元素的先後順序。時間複雜度為$O(N!)$
\end{enumerate}
% \subsubsection{爬行法}
\subsubsection{折半枚舉}
有時遇到複雜度$O(2^n)$的算法,在無法用其他方法降低複雜度,可以試著將元素切成兩半,降低n,再用其他算法組合起來。
\subsubsection{題目}
\begin{enumerate}
\item UVa 11059(區間列舉)
\item UVa 1481(區間列舉)
\item UVa 10976(減少列舉範圍)
\item UVa 750(回朔)
\item UVa 524(回朔)
\item UVa 11464
\item UVa 1326(折半枚舉)
\end{enumerate}
\subsection{貪心}
對於一個問題,始終使用同一種方法,採取在目前狀態下最好或最佳(即最有利)的選擇。\\
有的貪心很直觀,有的就需要通靈才解得出來,往往做題目一開始想到的辦法是錯的,直到做到一半才發現。所以我們需要證明方法是不是對的,這往往需要時間練習,才不會到比賽遇到時,花了很多時間去解題。\\
\subsubsection{證明的辦法}
\begin{enumerate}
\item 試圖構造出反例,發現他不存在。
\item 如果存在更佳解的答案比你做出來的還好,那這組解一定可以再做得更好,進而達到反證出更佳解不存在。
\item 使用遞迴證法:(1) 證明基底是對的。(2) 假設小問題是好的。(3) 你一定可以用最好的方法來將問題簡化成剛才假設是好的小問題。
\end{enumerate}
\subsubsection{題目}
\begin{enumerate}
\item UVa 11729
\item UVa 11292
\item UVa 11389
\item UVa 1445
\item UVa 993
\item TIOJ 1441
\end{enumerate}
\subsection{二分搜}
對於一個函數$F(n)$,如果存在一個x,對於所有 $\geq x$ 的a,$F(a)=$ true,反之$F(a)=$ false,基於這樣的單調性,就可以用二分搜。
\lstinputlisting{BasicA/binarySearch.cpp}
有些題目為"最多/最少為何會成立",那麼如果你可以在良好的時間檢查出"如果代價是 x,那可不可以達成目標",並且x具有單調性,那麼你可以轉換成"如果代價是 x,那可不可以達成目標"傳換成$F(x)$,對答案(x)進行二分搜。\\
二分搜要注意兩件事,一個是無限迴圈,要避免它可以在腦中先模擬一下。一個是在實數中二分搜,因為實數的稠密性,題目會有誤差容忍(例如$10^{-6}$),只要在誤差內都是容許的。
\subsubsection{三分搜}
對於U型函數(例如$y=F(x)=x^2$),我們想要找尋其極值,意謂其左右兩側皆各自遞增/遞減,我們可以利用三分搜來解決(二分搜只能解決全體單調性,不能解決有兩邊的)。\\
考慮三分後從左到右四個採樣點的關係
\begin{enumerate}
\item $S(a) < S(b) < S(c) < S(d)$,此時最小值一定不在最右邊
\item $S(a) > S(b) < S(c) < S(d)$,此時最小值一定不在最右邊
\item $S(a) > S(b) > S(c) < S(d)$,此時最小值一定不在最左邊
\item $S(a) > S(b) > S(c) > S(d)$,此時最小值一定不在最左邊
\end{enumerate}
這段描敘還可以再簡化
\begin{enumerate}
\item $S(b) < S(c)$,此時最小值一定不在最右邊
\item $S(b) > S(c)$,此時最小值一定不在最左邊
\end{enumerate}
每次都至少可以讓區間縮小$\frac{1}{3}$
\lstinputlisting{BasicA/3Search.cpp}
\subsubsection{題目}
\begin{enumerate}
\item Uva 714
\item Uva 1421
\item Uva 11627
\item zerojudge d732
\item neoj 72(三分搜)
\end{enumerate}
\subsection{分治}
分治法會把問題分解成子問題(分),解決完再合併回原本的問題(治)。\\
分治分成以下步驟
\begin{enumerate}
\item 切割:把一個問題切成子問題然後遞迴
\item 碰底:碰到不能再切割或是明顯有答案(也許無解),就算出答案再回傳
\item 合併:利用傳回來的子問題算出答案然後回傳
\end{enumerate}
\subsubsection{合併排序法}
一個利用分治實作的排序法,逆序數對也會利用他的概念來實作。
\begin{enumerate}
\item 切割:把序列分成兩半然後遞迴
\item 碰底:直到序列長度為1,這時候已為一個排好的序列,直接回傳
\item 合併:利用傳回來的兩串序列進行排序
\end{enumerate}
\lstinputlisting{BasicA/mergesort.cpp}
\subsubsection{更多的經典題目}
\begin{enumerate}
\item 快速排序法。
\item 逆序數對。(經典問題,搭配 Merge Sort)
\end{enumerate}
\subsubsection{題目}
\begin{enumerate}
\item uva 1608
\item uva 10810(逆序數對)
\item uva 11129
\item uva 10245
\end{enumerate}
