\section{基礎資料結構}
\subsection{什麼是STL?}
標準函式庫(Standard Template Libiary),C++內建的資料結構。
\subsection{型態模板}
當你要使用容器時,你必須要告訴C++說,你的資料型態是什麼,型態模板的用途就是在於此。\\
用法:$C <T> name$\\
而容器內部東西不會只有一個,像map就需要兩種型態。\\
$map <T1, T2> name$\\
有時候參數不須寫滿,不寫滿的地方的值為預設值。
\subsection{迭代器}
如果你想在容器中遍歷,你可能想用下標運算子`[]`,但不是所有容器都像陣列,都有支援下標運算子,所以C++為每個容器都提供一個資料型態叫"迭代器",你可以把迭代器當成一種指標,假設有一個迭代器it,加上星號`*`可以存取IT所指向的內容,依據迭代器的強到弱可分為三種:
\begin{enumerate}
\item 隨機存取(Random Access):可與整數做+-法、遞增及遞減
\item 雙向(Bidirectional)迭代器:遞增及遞減
\item 單向(Forward)迭代器:只能遞增
\end{enumerate}
根據用法可分為兩種:
\begin{enumerate}
\item 輸入(Input)迭代器:讀取迭代器指向的內容,,所有的迭代器都可以當作輸入迭代器。
\item 輸出(Output)迭代器:更改迭代器指向的內容時,除了常數(const)迭代器(也就是規定不能更動迭代器指向的內容)以外,所有的迭代器都可以當作輸出迭代器。
\end{enumerate}
C++在許多容器中提供正向和逆向迭代器,前者由前往後,後著由後往前,宣告時分別為$C::iterator$及$ C::reverse\_iterator$,每種迭代器分別有一對迭代器代表頭尾,如下表,注意end系列指向該容器最後一項的後一項,不要對他做人和取值或修改。\\ \\
\begin{tabular}{|l|c|c|}
	\hline &正向&逆向 \\
	\hline 可改值&C.begin()\ C.end()&C.rbegin()\ C.rend()\\
	\hline 不可改值&C.cbegin()\ C.cend()&C.crbegin()\ C.crend()\\
    \hline
\end{tabular}
\subsection{stack 堆疊}
有兩個端口,其中一個封閉,另一個端口負責插入、刪除的資料結構
\lstinputlisting{BasicDS/stack.cpp}
\begin{enumerate}
\item 標頭檔:<stack>
\item 建構式:stack <T> s
\item s.push(T a):插入頂端元素,複雜度$O(1)$
\item s.pop():刪除頂端元素,複雜度$O(1)$
\item s.top():回傳頂端元素,複雜度$O(1)$
\item s.size():回傳元素個數,複雜度$O(1)$
\item s.empty():回傳是否為空,複雜度$O(1)$
\end{enumerate}
\lstinputlisting{BasicDS/stack2.cpp}
\subsection{queue 佇列} 
有兩個端口,一個負責插入,另一個端口負責刪除的資料結構
\lstinputlisting{BasicDS/queue.cpp}
\begin{enumerate}
\item 標頭檔:<queue>
\item 建構式:queue <T> q
\item q.push(T a):插入尾端元素,複雜度$O(1)$
\item q.pop():刪除頂端元素,複雜度$O(1)$
\item q.front():回傳頂端元素,複雜度$O(1)$
\item q.size():回傳元素個數,複雜度$O(1)$
\item q.empty():回傳是否為空,複雜度$O(1)$
\end{enumerate}
\lstinputlisting{BasicDS/queue2.cpp}
\subsection{deque 雙向佇列} 
有兩個端口,皆負責刪除、插入的資料結構
\begin{enumerate}
\item 標頭檔:<deque>
\item 建構式:deque <T> dq
\item dq.push\_front(T a),dq.push\_back(T a):插入頂端/尾端元素,複雜度$O(1)$
\item dq.pop\_front(),dq.pop\_back():刪除頂端/尾端元素,複雜度$O(1)$
\item dq.front(),dq.back():回傳頂端/尾端元素,複雜度$O(1)$
\item dq.size():回傳元素個數,複雜度$O(1)$
\item dq.empty():回傳是否為空,複雜度$O(1)$
\end{enumerate}
\subsection{list}
陣列如果要從中間插入一個元素,需要將其後面所有元素搬移一格,需耗費$O(n)$,連結串列(linklist)能只花$O(1)$完成插入。
\lstinputlisting{BasicDS/linklist1.cpp}
\lstinputlisting{BasicDS/linklist2.cpp}
但有時候我們也可以利用idnex取代指標來實作linklist,兩種做法各有自己的優缺點。\\
C++提供list函式庫實作雙向串列
\begin{enumerate}
\item 標頭檔:`<list>`
\item 建構式:`list <T> L`
\item L.size():回傳元素個數,複雜度$O(1)$
\item L.empty():回傳是否為空,複雜度$O(1)$
\item L.push\_front(T a),L.push\_back(T a):插入頂端/尾端元素,複雜度$O(1)$
\item L.pop\_front(),L.pop\_back():刪除頂端/尾端元素,複雜度$O(1)$
\item L.insert(iterator it,size\_type n,T a):在 it 指的那項的前面插入 n 個 a 並回傳指向 a 的迭代器。複雜度 $O(n)$。
\item L.erase(iterator first,iterator last):把 $[first,last)$ 指到的東西全部刪掉,回傳 last。複雜度與砍掉的數量呈線性關係,如果沒有指定 last,那會自動視為只刪除 first 那項。
\item L.splice(iterator it,list\& x,iterator first,iterator last):first 和 last 是 x 的迭代器。此函式會把 $[first,last)$ 指到的東西從 x 中剪下並加到 it 所指的那項的前面。x 會因為這項函式而改變。若未指定 last,那只會將 first 所指的東西移到 it 前方。複雜度與轉移個數呈線性關係。
\end{enumerate}
\lstinputlisting{BasicDS/list.cpp}
\subsection{array}
更安全方便的陣列
\begin{enumerate}
\item 標頭檔:<array>
\item 建構式:array <T,N> a
\item a.size():回傳元素個數,複雜度$O(1)$
\item a.fill(T val):將每個元素皆設為val,複雜度$O(size)$
\end{enumerate}
\lstinputlisting{BasicDS/array.cpp}
\subsection{vector}
動態陣列,可改變長度
\begin{enumerate}
\item 標頭檔:<vector>
\item 建構式:vector <T> v
\item v[i]:回傳v的第i個元素,複雜度$O(1)$
\item v.push\_back(T a):插入尾端元素,複雜度$O(1)$
\item v.size():回傳元素個數,複雜度$O(1)$
\item v.resize():重設長度,複雜度$O(1)$
\item v.clear():清除元素,複雜度$O(size)$
\item 兩特化
    \begin{enumerate}
    \item vector<char>:string
    \item vector<bool>:bitset
    \end{enumerate}
\end{enumerate}
\lstinputlisting{BasicDS/vector.cpp}
\subsection{string} 
可變動長度的字元陣列
\begin{enumerate}
\item 標頭檔:<string>
\item 建構式:string s
\item s=t:讓s的內容和t一樣,複雜度通常是$O(size(s)+size(t))$
\item s+=t:將t加到s後面,複雜度通常是$O(size(s)+size(t))$
\item s.cstr():回傳和s一樣的C式字串,複雜度$O(1)$
\item cin>>s:輸入字串至s,直到讀到不可見字元
\item cout<<s:輸出字串s
\item getline(cin,s,char c):輸入s直到遇到字元c,未指定c時,c為預設字元。
\end{enumerate}
\lstinputlisting{BasicDS/string.cpp}
\subsection{bitset} 
節省的bool陣列,可當二進位位元運算
\begin{enumerate}
\item 標頭檔:<bitset>
\item 建構式:bitset <N> b(a)
\item b[a]:存取第 a 位,複雜度 $O(1)$。
\item b.set():將所有位元設成1。複雜度 $O(N)$。
\item b.reset():將所有位元設成 0。複雜度 $O(N)$。
\item b(位元運算):不管是一元、二元運算皆可。二元位元運算長度需一致。複雜度 $O(N)$。
\item b.count():回傳 b 有幾個位元是 1。複雜度 $O(N)$。
\item b.flip():將所有位元的 0、1 互換。複雜度 $O(N)$。
\item b.to\_string():回傳一個字串和 b 的內容一樣。複雜度 $O(N)$。
\item b.to\_ulong():回傳一個 unsigned long 和 b 的內容一樣 (在沒有溢位的範圍內)。複雜度,$O(N)$。
\item b.to\_ullong():回傳一個 unsigned long long 和 b 的內容一樣 (在沒有溢位的範圍內),複雜度 $O(N)$。
\end{enumerate}
\subsection{priorty\_queue 優先佇列} 
維護最大/小值,可插入、刪除、及詢問最大/小值,一種實作為binary heap
\lstinputlisting{BasicDS/heap.cpp}
\begin{enumerate}
\item 標頭檔:<queue>
\item 建構式:priorty\_queue <T> pq
\item 建構式:priorty\_queue <T,Con,Cmp> pq
\item 建構式:priorty\_queue <T,Con,Cmp> pq(iterator first, iterator seecond)插入$[first,second)$內的東西
\item pq.push(T a):插入元素a,複雜度$O(\log size)$
\item pq.pop():刪除頂端元素,複雜度$O(\log size)$
\item pq.top():回傳頂端元素,複雜度$O(1)$
\item pq.size():回傳元素個數,複雜度$O(1)$
\item pq.empty():回傳是否為空,複雜度$O(1)$
\end{enumerate}
\lstinputlisting{BasicDS/PQ.cpp}
\subsection{pair} 
兩個資料型態組織合盟
\begin{enumerate}
\item 標頭檔:`<utility>`
\item 建構式:`pair <T1,T2> p`
\item p.first,p.second:p的第一、二個值
\item make\_pair(T1 a1,T2 a2):回傳一個(a1,a2)的pair
\end{enumerate}
\subsection{tuple} 
generalize的pair
\begin{enumerate}
\item 標頭檔:`<tuple>`
\item 建構式:tuple<T1,T2...> t
\item `get<i>(t)`:回傳 t 的第 i 個值。
\item make\_tuple(T1 a1,T2 a2,...):回傳一個 (a1,a2,...)的tuple。
\end{enumerate}
\subsection{set/map 自查找平衡二元樹} 
\begin{enumerate}
\item set和map支援插入、刪除及查詢一個值,不同的是,set會 
\item 也可以說set的鍵值和對應值一樣
\end{enumerate} 
\subsection{set} 
\begin{enumerate}
\item 標頭檔:`<set>`
\item 建構式:`set <T1> s`
\item s.size():回傳元素個數,複雜度$O(1)$
\item s.empty():回傳是否為空,複雜度$O(1)$
\item s.clear():清除元素,複雜度$O(size)$
\item s.insert(T1 a):加入元素 a,複雜度 $O(\log size)$。
\item s.erase(iterator first,iterator last):刪除 $[first,last)$,若沒有指定 last 則只刪除 first,複雜度$O(\log size)$與加上元素個數有關係。
\item s.erase(T1 a):刪除鍵值 a,複雜度 $O(\log size)$。
\item s.find(T1 a):回傳指向鍵值 a 的迭代器,若不存在則回傳 s.end(),複雜度 $O(\log size)$。
\item s.lower\_bound(T1 a):回傳指向第一個鍵值大於等於 a 的迭代器。複雜度 $O(\log size)$。
\item s.upper\_bound(T1 a):回傳指向第一個鍵值大於 a 的迭代器。複雜度 $O(\log size)$。
\end{enumerate}
\lstinputlisting{BasicDS/set.cpp}
\subsection{map}
\begin{enumerate}
\item 標頭檔:`<map>`
\item 建構式:`map <T1, T2> m`
\item m.size(),m.empty(),m.clear(),m.erase(iterator first,iterator last),m.erase(T1 a),m.find(T1 a),m.lower\_bound(T1 a),m.upper\_bound(T1 a):同 set。
\item `m[a]`:存取鍵值 a 對應的值,若 a 沒有對應的值,會插入一個元素,使 a 對應到預設值並回傳之。複雜度 $O(\log size)$。
\item m.insert(pair<T1,T2> a):若沒有鍵值為 a.first 的值,插入一個鍵值為 a.first 的值對應到a.second,並回傳一個 pair,first 是指向剛插入的元素的迭代器、second 是 true;若已經有了,回傳一個 pair,first 是指向鍵值為 k.first 的元素的迭代器,second 是 false。複雜度 $O(\log size)$。
\lstinputlisting{BasicDS/map.cpp}
\end{enumerate}
\subsubsection{multi-系列}
可插入重複元素代價為map無法用下標運算子
\begin{enumerate}
\item equal\_range(T1 a):回傳iterator的pair<lower\_bound(a),upper\_bound(a)>,為a所在範圍
\item erase(T1 a):刪除所有元素a,如果只要刪除一個,用s.erase(s.find(a))
\end{enumerate}
\subsubsection{unorder\_系列}
降低常數,期望複雜度少一個 log,代價為不會排序,沒有lower\_bound/upper\_bound,也不會依鍵值大小遍歷。迭代器為單向。
\subsection{題目}
\subsubsection{stack,queue,deque}
先來看一題題目\\ \\
(zjd555)
在平面上如果有兩個點 (x,y) 與 (a,b),我們說 (x,y) 支配(Dominate)了(a,b)這就是指 x $\geq$ a
而且 y $\geq$ b;用圖來看就是 (a,b) 座落在以 (x,y) 為右上角的一點無的區域中。\\ 
對於平面上的任意一個有限點集合而言,一定存在有若干個點,它們不會被集合中的內一點所支配,這些個數就構成一個所謂的極大集合。請寫一個程式,讀入一個新的集合,找出這個集合中的極大值。\\
簡單的說  若找不到一點在 (x,y) 的右上方,則 (x,y) 就要輸出\\ \\
我們能肯定最右上角的點一定是極大點,再來的極大點一定都在他的右下方,那我們就以座標排序,先以y座標由大到小排序,再以x座標由大到小排序,每找到一個極大點,就找下一個y值比它小,x比它到的點,那也是極大點,就這樣以此類推。\\
我們觀察到極大集合中,如果先以Y值排序由大到小排序,X值會由小到大,如果我們發現,答案的候選人,有如果...就大於/小於...的關係,我們就稱答案有"單調性"。單調性這個東西筆者覺得很玄,我沒辦法明確定義什麼事單調性,需要多多寫題目才能自己感覺出來。\\
下列提供stack,queue,deque的題目,前半部分是基礎應用,後半部分是關於單調性的。
\begin{enumerate}
\item UVa514(Stack 應用)
\item UVa673(Stack 應用)
\item ZeroJudge d016(Stack 後續運算法)
\item UVa10935(Queue 應用)
\item UVa12100(Queue 應用)
\item Uva246(Deque 應用)
\item UVa11871(stack 單調)
\item TIOJ1618(Deque 單調)
\end{enumerate}
\subsubsection{list}
\begin{enumerate}
\item ZeroJudge d718
\item TIOJ 1225
\item TIOJ 1930
\end{enumerate}
\subsubsection{string}
\begin{enumerate}
\item ZeroJudge a011(getline 應用)
\item Zerojudge d098(StringStream 應用,請自行查詢)
\end{enumerate}
\subsubsection{PriortyQueue}
\begin{enumerate}
\item Uva 10954
\item Uva 1203
\end{enumerate}
\subsubsection{set,map}
\begin{enumerate}
\item ZeroJudge d512(Set 應用)
\item UVa 10815(Set 應用)
\item Zerojudge d518(Map 應用)
\item UVa 484(Map 應用)
\end{enumerate}









