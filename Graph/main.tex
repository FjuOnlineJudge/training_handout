\section{Graph}
圖是由邊集合和點集合所形成的圖形,這種圖形通常用來描述某些事物之間的某種特定關係。頂點用於代表事物,連接兩頂點的邊則用於表示兩個事物間具有這種關係。\\
數學式為$G=(V,E)$。$G$代表圖(Graph),$V$代表點(vertex),$E$代表邊(edge)。
\subsection{術語}
在深入圖論中,我們先介紹一些術語,這些術語在後面的內容,不時會扮演著關鍵角色。
\begin{enumerate}
\item 無向邊、有向邊:邊具有方向性,無向邊代表邊沒有指定方向,$(u,v)$和$(v,u)$等價;有向邊則有指定方向,$(u,v)$和$(v,u)$是不同的。
\item 無向圖、有向圖、混合圖:無向圖是只有無向邊的圖,類似地,有向圖是只有有向邊的圖,混和圖則是包含無向邊和有向邊。
\item 點、邊個數:一般來說會用$|V|$、$|E|$來表示,在表示複雜度時,為了方便會用$V$、$E$來表示
\item 權重(weight):在點或邊上附帶一個數字稱做"權重",邊上權重較常見,權重通常代表代價,例如所需花費時間或金錢。
\item 相鄰 (adjacent):無向圖中,兩個點 $u$, $v$ 相鄰代表存在一個邊 $e_i = (u, v)$。
\item 指向 (consecutive):有向圖中, $u$ 指向 $v$ 代表存在一個邊 $e_i = (u, v)$。
\item 度(degree):無向圖中,一個點連到的邊數稱為"度",在有向圖稱為出度(out-degree,簡稱 $d_{out}$)及入度(in-degree,簡稱 $d_{in}$),分別代表該點指向別點及被指向的邊數。
\item walk:一條由$x$到$y$的路徑$x=v_1,v_2,v_3...,v_k=y$。
\item trail:一條不重複邊的walk。
\item 迴路(circut):起點和終點一樣的trail。
\item path:一條不重複點(起點和終點例外)的walk。
\item 環(cycle):起點和終點一樣的path。
\item 自環 (loop):一條邊 $e_i = (u, v)$ 滿足 $u = v$,$e_i$ 即稱為自環。
\item 重邊 (multiple edge):在一張圖中,存在 $e_i$, $e_j$ 滿足 $i$ != $j$ and $e_i = e_j$,則稱為重邊。
\item 連通 (connected):無向圖中,若 $u$ 和 $v$ 存在路徑,則 $u$ 和 $v$ 連通。若一群點兩兩連通,則這些點都連通。
\end{enumerate}
\subsection{特殊圖}
有一些特殊的圖,擁有一的特殊的性質可以幫助解題,或是少到一些麻煩的條件讓題目好處理。
\begin{enumerate}
\item 簡單圖(Simple Graph) 沒有自環和重邊的圖。
\item 連通圖(connected Graph) 無向圖中,任意兩點皆可經過一些邊訪問彼此,這張圖即為無向圖。
\item 樹(Tree) 一張沒有環且連通的圖。
\item 森林(Forest) 由無數個互不連通的樹所形的圖為森林。
\item 完全圖(Complete Graph) 無向圖中,任意兩點$u, v$皆存在一條邊$e_i = (u, v)$,稱為完全圖。一張n個點的完全圖簡記為$K_n$,在集合上曾為完全圖為"團"
\item 競賽圖(Tournament Graph) 有向圖中,任意兩點$u, v$皆存在一條邊$e_i = (u, v)$,稱為競賽圖。
\item 有向無環圖(Directed acyclic graph, DAG) 沒有環的無向圖。
\item 二分圖(Bipartite Graph) 能將圖上的點分成兩個集合,任意一條邊$e_i = (u, v)$都滿足,$(u, v)$在不同集合裡,該圖稱為二分圖。
\item 平面圖(Planar Graph) 可畫在平面上,且任意兩條邊皆不重疊的圖。
\end{enumerate}
\subsection{圖的關係}
\begin{enumerate}
\item 子圖(subgraph) 如果$G'=(V', E')$是$G=(V, E)$的子圖,則$V'\in V$且$E'\in E$。
\item 補圖 (complement graph) graph):令 G = (V, E) 是一個圖,K 包含所有 V 的二元子集 (2-element subset)。則圖 H = (V, K\\+E) 是 G 的補圖。換句話說,把原本的邊移除,加入原本不存在的邊即是補圖。
\item 同構 (isomorphic)
\end{enumerate}
\subsection{儲存}
至於圖要怎麼存起來呢,以下介紹兩種辦法。設$V$為點數,$E$為邊數。
\subsubsection{相鄰矩陣(adjacency matrix)}:開一個 $V \times V$ 的資料結構$M$(通常會用二維陣列),$M[a][b]$代表的是點$a$至$b$的邊數或權重。空間複雜度$O(V^2)$。加、刪邊時間複雜度 $O(1)$。
\subsubsection{相鄰串列(adjacency list)}:開$V$個可變長度的資料結構 (通常在 C++ 用 vector、在C用 linked list),第$i$個裡面放所有第$i$個點指向的點的編號 (和邊權或其他邊的資訊)。空間複雜度$O(V + E)$,加邊時間複雜度$O(1)$、刪邊時間複雜度$O(V)$。
至於使用時機,如果邊數較密,且頻繁地需要找尋兩點之間的權重,那麼相鄰矩陣比較適合,其餘情況則是用相鄰串列。
\subsection{遍歷}
存好圖後,為了獲得某些資訊,需要一些讀取的方法,這些方法我們叫做"遍歷"。以下介紹兩種方法:DFS和BFS。
\subsubsection{DFS}
找到一條新的路就繼續找下去,直到沒有新的路時,原地返回。通常用遞迴實作或用stack維護。
\lstinputlisting{Graph/DFS.cpp}
\subsubsection{BFS}
\lstinputlisting{Graph/BFS.cpp}
把所有看到的路都加入清單中,並且以加入的順序來遍歷。通常以queue來維護。\\
BFS和DFS的時間複雜度皆為$O(V+E)$,空間複雜度皆為$O(V)$。\\
注意,圖不一定完全連通,我們通常會另外開一個陣列(bitset)紀錄是否拜訪過。
\subsubsection{題目}
\begin{enumerate}
\item zerojudge a290(給你一張有向圖,問你可不可以由A走到B)
\item zerojudge a982(二維迷宮問題)
\item zerojudge a634(馬步問題)
\item UVa 572
\end{enumerate}
\subsection{樹}
樹是一種特殊的圖,有許多算法都是由樹發展出來。
\subsubsection{特性}
以下這些無向圖的敘述都是在表示樹,這些敘述在競賽中有時能引導出答案。
\begin{enumerate}
\item 為連通圖且$|V|=|E|+1$
\item 任意兩個點之間存在唯一路徑
\item 為連通圖,但拔掉一條邊即為不連通(分成兩張連通圖)。
\item 沒有環,但加上一條邊會形成環。
\item 若節點編號存在順序,除了第一個節點,每個節點都會伸出一條邊連到順序比自己前面的節點。
\end{enumerate}
\subsubsection{術語}
樹同樣也有一些術語要知道的。
\begin{enumerate}
\item 根(root):樹的一個代表性的點,通常會被當遍歷的起點,有給定根點的樹叫"有根樹"。至於無根樹依照題目需求,有時要隨機找一個點當根。
\item 葉解點(leaf):度數為1的節點,有根樹的根結點則會是題目需求來決定是否為葉節點。
\item 父節點、子節點:有根樹中,兩個相連的節點,比較接近樹根的為父節點,反之為子節點。
\item 祖先(ancestor)、子孫(descandent):有根樹中,節點到根結點中,所有的節點皆為祖先。反過來說,該節點是他的祖先的子孫。依題目所需,有時自己也是自己的祖先(尤其是根最常這樣定義)。
\item 距離( distance):為兩個點所形成路徑之邊數,或是路徑上權重之和。
\item 深度(depth):有根樹中,節點到根結點之距離。
\item 高度(height):有根樹中,節點到與它距離最大的葉節點的距離稱為高度。根的高度稱為這整顆樹的高度。
\item 子樹(subtree):設有兩棵樹$T$,$T_1$,如果$V_1\in V$,$E_1\in E$,那麼我們說 $T_1$ 為 $T$ 的子樹
\item N元樹:每個節點最多有N個節點,稱為N元樹。
\end{enumerate}
\subsubsection{時間戳記}
如果我們從樹根DFS,紀錄第一次經過(進入)和最後一次經過(離開)的時間,這種技巧稱為"時間戳記"(time stamp)。
\lstinputlisting{Graph/timestamp.cpp}
有了時間戳記,我們有以下性質:
對於樹上任意兩點$u,v$,$u$是$v$的祖先 $iff$ $t_{in}(u) \leq t_{in}(v)$且$t_{out}(v) \leq t_{out}(u)$,此性質還可以推出,一顆子樹必在一個連續的區間,於是時間戳記就可以搭配其他資料結構,來解決樹上的RMQ問題了(這裡先不提)。
\subsubsection{最低共同祖先(Lowest Common Ancestor, LCA)}
在有根樹上任意兩點$u,v$,LCA(u,v)是指,兩點祖先交集中,深度最深的一個點。暴力尋找LCA花$O(V)$時間,不太理想,因此有很多演算法發評出來解決這問題,倍增法是最常見之一。\\
倍增法是利用DP是做出來,$par[v][i]$代表$v$的第$2^i$層祖先。
\lstinputlisting{Graph/doubling.cpp}
時間複雜度為$O(V\log V)$。\\
建好表之後,由於兩點的共同祖先有單調性,以$u$點的所有祖先要看,在LCA(包含)以上的祖先也會是兩點的共同祖先,否則就只是$u$的祖先,,因此可以用二分搜來尋找LCA了。
\lstinputlisting{Graph/LCA.cpp}
時間複雜度為$O(\log V)$。\\
有了LCA我們可以找出$u, v$的(唯一一條)路徑,也可以找出該路徑長度、路徑上最小(大)權重的邊...,在處理樹的問題,LCA是一大利器。
\subsubsection{題目}
\begin{enumerate}
\item UVa 1357(時間戳記)
\item zerojudge d767(LCA)
\end{enumerate}
\subsection{二元樹}
二元樹在程式競賽中常常被討論,有許多資料結構都是二元樹,例如STL提到的heap。
\subsubsection{遍歷}
基於根結點遍歷順序,二元樹的遍歷有"前序、中序、後序"三種辦法,通常DFS會使用前序來實作,如果想要得到任何一種順序的結果,只要改變輸出的順序。
\lstinputlisting{Graph/binarytreeorder.cpp}
\subsubsection{二元搜尋樹(Binary Search Tree, BST)}
二元搜尋樹是二元樹的應用,利用遞迴方式來定義,如下:
\begin{enumerate}
\item 根結點的值大於左子節點的值,小於右子節點的值。
\item 其左右子樹亦為二元搜尋樹。
\end{enumerate}
用上敘定義,就可以建造出BST,不過如果我們將一個以排序的串列建成BST,會發現BST會"退化"成一條傾斜的串列。BST本身不實用,重要在於它的推廣結構,例如AVL樹、紅黑樹、treap,不過這些資料結構比較進階,在這裡先不提。
\subsection{並查集}
並查集是一種樹狀結構,他支援兩件事
\begin{enumerate}
\item 查詢所隸屬集合
\item 合併兩個集合
\end{enumerate}
我們把集合轉化成樹,一顆樹代表一個集合,樹根代表集合的老大,查詢隸屬集合就回傳樹根是誰(一個樹餔可能有兩顆樹根吧),合併的時侯,就把一顆樹的樹根只到另一顆,以下為詳細的描述。
\subsubsection{初始}
一開始的時候,每個點自成一個集合,所以把樹根都設為自己。
\subsubsection{查詢}
查詢的時候,要查到樹根為自己的點,為止否則的話就要繼續查。
\subsubsection{狀態壓縮}
在合併之後原本被指向的樹根就沒用了,我們可以一邊做查詢時,一邊做更新。
\subsubsection{啟發式合併}
建立一個$h[i]$代表樹的高度,亦是元素最大遞迴次數,$h[i]$一開始為1。再來,我們每次都讓高度小的高度大的合併,如果遇到高度一樣的,就讓合併別人的樹高度加1。如果要把高度變為$x$,則至少需要$2^x$個點,由此推出N個點所形成最高之高度為$\log(N)$。\\
\lstinputlisting{Graph/DisjointSet.cpp}
複雜度為$O(\alpha(N))$。並查集最常用的地方是最小生成樹的Kruskal’s algorithm。
\subsubsection{題目}
\begin{enumerate}
\item zerojudge d808
\item UVa 1160
\item UVa 10158(陣列開兩倍)
\item UVa 1329(帶權並查集)
\end{enumerate}
\subsection{最小生成樹(Minimun Spanning Tree, MST)}
在一張圖中,如果有子圖剛好為也為一顆樹,我們就稱該子圖為生成樹。現在我們在圖上加上權重,而在所有的生成樹中,權重最小的,我們稱為"最小生成樹",最小生成樹並不唯一,以下介紹幾種最小生成樹的演算法。
\subsubsection{Kruskal’s algorithm}
Kruskal’s algorithm的概念是,合併兩顆MST的時候,加入連接兩顆樹中,最小權重的邊。所以我們就利用greedy,將邊依權重由小到大排序,如果邊的兩邊是在不同的MST,我們就把合併(並查集應用於此),反之就跳過。排序需花$O(E\log E)$的時間,選邊需要花$O(E\alpha(V))$的時間,總共時間複雜度$O(E(\log E+\alpha(V)))$
\lstinputlisting{Graph/Kruskal.cpp}
\subsubsection{Prim’s algorithm}
Prim’s algorithm的思維則是,將一棵MST連出的邊中,加入權重最小的邊(距離最近的點),重複執行後得出最小的生成樹。在實作上,首先取一個點當MST,更新所有與它相鄰的點,更新後把距離最小的點加入MST(不用並查集),持續執行更新及加入點的動作,直到所有點都已加入MST。維護最小距離用priority\_queue維護,每個點只會被合併一次,每條邊都只會遍歷一次,複雜度$O((V+E)logE)$。另外有一個資料結構用費波那契堆(fibonacci heap)可以達到 $O(E+V\log V)$。但是因為它常數比較大,實作複雜,我們不會使用它。總體而言,Kruskal比Prim好用。
\subsubsection{Borůvka’s algorithm}
Borůvka’s algorithm和Prim一樣都在加入MST和最鄰近的點,不一樣的是,它讓所有的MST一起做這件事。每次找出每棵MST外權重最小的邊,並加入MST(如果權重一樣,就找索引值最小的),檢查是否只剩一棵MST,如果不是就重複掃描的動作,這裡一樣用並查集維護聯通性。\\
最差的情況為每次都剛好兩兩成對合併,這樣最多只會執行$\log V$次,整題複雜度為$O((V+E)\log V)$)。期望複雜度可以達到$O((V+E))$ (因為每次並查集都會被合併+查詢,所以$\alpha$可以完全省略)。
\subsubsection{題目}
\begin{enumerate}
\item zerojudge a129
\end{enumerate}
\subsection{最短路徑}
\subsubsection{術語}
\begin{enumerate}
\item 負邊:權重為負的邊
\item 負環:權重和為負的環
\item 點源:成為起點的點,分成單源頭及多源頭。
\item 鬆弛:單源頭最短路徑中,對於任意兩個點$u,v$,起點$s$到它們的距離$d_u,d_v$,如果$d_u>d_v+w_{u,v}$,$w_{u,v}$為邊$(u,v)$的權重,我們可以讓$d_u$更新為$d_v+w_{u,v}$,讓$s$到$u$的距離縮短,這個動作稱為"鬆弛"。 
\end{enumerate}
\subsubsection{Floyd-Warshall Algorithm}
為多源頭最短路徑,求出所有點對的最短路徑。\\
Floyd-Warshall是動態規劃,以下是他的dp式。
\begin{enumerate}
\item 狀態:$dp[k][i][j]$ 代表,若只以點 $1 ∼ k$ 當中繼點的話,則 $dp[k][i][j]$ 為 $i$ 到 $j$ 的最短路徑長。
\item 轉移:$dp[k][i][j] = min(dp[k − 1][i][k] + dp[k − 1][k][j], dp[k − 1][i][j])$
\item 基底:$dp[0][i][j] = \left\{ \begin{array}{cc}
w[i][j] & if\ w[i][j]\ exists\\
INF & else
\end{array} \right\}$
\end{enumerate}
時/空間複雜度皆為$O(V^3)$,然而此DP是可以滾動,所以空間複雜度可降為$O(V^2)$\\
\lstinputlisting{Graph/FW.cpp}
執行的時候如果$dp[i][j]<0$,代表存在負環,Floyd-Warshall是可以判斷負環。
\subsubsection{單點源最短路徑}
求出一個點到所有點的最短路徑,其實就是以起點為根,最短路徑是由父節點鬆弛而來的最短路徑樹。我們找最短路徑,就是一直把鬆弛,直到所有點都不能鬆弛,所有點都獲得最短路徑了。要蓋出最短路徑樹,就只要把點指向最後一次被誰鬆弛就好了。
\subsubsection{Bellman-Ford Algorithm}
為單點源最短路徑,設起點的最短路徑為0,其他點為無限大,每次對所有邊枚舉,因為最短路徑不會經過同樣的邊第二次,所以只要執行$V-1$輪,複雜度為$O(VE)$。如果執行第V次時還有邊可以鬆弛,代表有負環,Bellman-Ford也可以當成負環的判斷方法。\\
\lstinputlisting{Graph/bellmanford.cpp}
此演算法還有一個優化版本叫做Shortest Path Faster Algorithm(SPFA),他的做法是枚舉起點是鬆弛過的邊,以鬆弛過的點除非被重新鬆弛,否則不會更動。預期複雜度為$O(V+E)$,不過最差狀況仍為$O(VE)$。
\subsubsection{Dijkstra’s Algorithm}
同樣為單點源最短路徑,他的想法和Prim's Algorithm類似,每次把離樹根最近的點加入最短路徑樹裡,並把所有與該點相連的邊鬆弛,已經加入的點不會在被鬆弛。使用priority\_queue的複雜度為$O((V+E)\log E)$,使用費波那契堆,複雜度為s$O(E+V\log V)$
\lstinputlisting{Graph/dijkstra.cpp}
而Dijkstra’s Algorithm不能處理負邊,原因是一旦點加入最短路徑樹,就不會再被更新,以維持良好複雜度,負邊會破壞此規則。
\subsubsection{題目}
\begin{enumerate}
\item UVa 534
\item UVa 10048
\item UVa 929(方格上)
\item UVa 11090
\end{enumerate}
\subsection{歐拉迴路}
柯尼斯堡七橋問題可說是圖論最早的起源,問題如下(from 維基百科):"當時東普魯士柯尼斯堡(今日俄羅斯加里寧格勒)市區跨普列戈利亞河兩岸,河中心有兩個小島。小島與河的兩岸有七條橋連接。在所有橋都只能走一遍的前提下,如何才能把這個地方所有的橋都走遍?"歐拉解決這個問題,圖論也因此誕生。
七橋問題根據起點與終點是否相同,分成Euler path及Euler circuit。
無向圖部分,將點分成奇點(度數為奇數)和偶點(度數為偶數)。
\begin{enumerate}
\item Euler path:奇點數為0或2
\item Euler circuit:沒有奇點
\end{enumerate}
證明from wiki:\\
必要性:\\如果一個圖能一筆畫成,那麼對每一個頂點,要麼路徑中「進入」這個點的邊數等於「離開」這個點的邊數:這時點的度為偶數。要麼兩者相差一:這時這個點必然是起點或終點之一。注意到有起點就必然有終點,因此奇頂點的數目要麼是0,要麼是2。\\
充分性:\\如果圖中沒有奇頂點,那麼隨便選一個點出發,連一個環 $C_{1}$。如果這個環就是原圖,那麼結束。如果不是,那麼由於原圖是連通的, $ C_{1}$ 和原圖的其它部分必然有公共頂點 $s_{1}$。從這一點出發,在原圖的剩餘部分中重複上述步驟。由於原圖是連通圖,經過若干步後,全圖被分為一些環。由於兩個相連的環就是一個環,原來的圖也就是一個歐拉環了。\\如果圖中有兩個奇頂點 $u$ 和 $v$,那麼加多一條邊將它們連上後得到一個無奇頂點的連通圖。由上知這個圖是一個環,因此去掉新加的邊後成為一條路徑,起點和終點是 $u$ 和 $v$。證畢。\\
有向圖部分,將點分成出點(出度-入度=1)和偶點(入度-出度=1)還有平衡點(出度=入度)。
\begin{enumerate}
\item Euler path:出點和入點個數同時為0或1。
\item Euler circuit:只有平衡點。
\end{enumerate}
找出一組解要使用DFS,以下為算法框架
\begin{enumerate}
\item [判斷] 無向圖判斷其點個數,如果=0,選任意一點DFS,如果=2,選任意一奇點DFS,否則無解。有向圖判斷出入點個數,如果只有平衡點,選任意一點DFS,如果出點=入點=1,由出點DFS,否則無解。
\item [DFS] 若當前節點還有尚未走過的邊,那麼拜訪該邊,並在拜訪完後輸出該邊,否則離開當前結點。若還有節點尚未拜訪,則無解,否則輸出順序即為一組解(無向圖)/將輸出順序反向(有向圖)即為一組解。
\end{enumerate}
\subsubsection{題目}
\begin{enumerate}
\item UVa 10054
\item UVa 10441
\end{enumerate}
\subsection{Hamilton問題}
跟歐拉迴路問題很像,不過這次不能重複的是點。至於判斷是否存在 Hamilton Circuit 、找到一個 Hamilton Circuit 是 NP-complete 問題,找到一個權重最小的 Hamilton Circuit 是 NP-hard 問題,目前尚未出現有效率的演算法。\\
現在用DP可以做到$O(2^n\times n^2)$的複雜度,不過因為我們還沒教到狀態壓縮DP所以先不說。
\subsection{拓撲排序}
拓樸排序是對將有向圖轉換成一個線性序列,滿足圖上任意一條邊$(u,v)$,在線性序列中,$u$在$v$之前。一個常見的應用學校的擋修機制,要修一門課必須修完他的先修課。\\
我們藉由擋修機制來轉換圖論,把課程轉為點,把有擋修關係的課程連一條邊,先修課指向後修課。如果把點(修完的課)拔掉,那麼當一個點沒有入度時(意即修完該課程的所有先修課),就可以被拔掉。\\
有兩種方法可以找出拓譜排序,一種是用queue,把所有入度$=0$的點都進去,每一次都從queue拿出一個點,將他所有指向的點入度都-1,其中如果有點的入度變成0,一樣將該點放進queue裡面,如果queue裡面沒點,但是還有點入度$>0$,則無解,否則從queue拿出的順序就是拓樸排序。\\
\lstinputlisting{Graph/TopologicalSort.cpp}
另一種是利用DFS+時間戳記,如果發現有任一條邊$(u,v)$,$tin[u]>tin[v]$,那就無解,否則依照$tout$由大到小形成拓譜排序。\\
這兩種辦法的時間複雜度皆為$O(V+E)$
\subsubsection{題目}
\begin{enumerate}
\item UVa 10305
\item UVa 1119
\end{enumerate}
\subsection{Connectivity}
一張任意兩點接連通的圖叫做連通圖,在實際情況,例如網路或電力的架設都希望線路是連通的,要是地方壞掉,我們希望影響能越小越好。在圖論中,有算法可以找出去掉那些部分會使得圖變成不連通的,以下詳細說明。
\subsubsection{DFS邊分類}
根據DFS的順序(時間戳記),對邊進行分類,這些分類在之後的章節會用到。
\begin{enumerate}
\item [Tree edge] 連到兒子的邊
\item [Back edge] 子孫連到祖先的邊
\item [Forward edge] 連到子孫(非兒子)的邊
\item [Cross edge] 連到非直系血親的邊
\end{enumerate}
其中有向圖是四種邊都有,無向圖只有前面兩種。
\subsubsection{無向圖的雙連通}
我們先來定義一些術語
\begin{enumerate}
\item [點連通度] 最少要移除多少個點才會讓整張圖不再連通
\item [邊連通度] 最少要移除多少條邊才會讓整張圖不再連通
\item [雙連通] 移除任意一個"x"後,圖依然是連通的,就稱為"x-雙連通"。依照"x"的不同,可分為"點雙連通"及"邊雙連通"。
\end{enumerate}
在之前提到的例子,網路的架設,需要特別注意雙連通的問題,萬一有部分的線路(線)或是設備(點)損壞,就有可能導致一部分的網路不連通。以下分別從點和線的角度探討雙連通。
\subsubsection{點雙連通}
要判斷一張圖是否點雙連通,就要檢查他是否有割點,如果沒有割點,則這張圖為點雙連通。
\begin{enumerate}
\item [割點] 給定一張圖$G$,如果移除點$v$及連接$v$的邊之後,圖$G$不再連通,點$v$都被稱為$G$的一個割點(cut-vertex)或關節點(articulation-vertex, articulation-point)。
\end{enumerate}
至於怎麼找出割點,我們用tarjan演算法。首先我們用DFS為無向圖建立一顆DFS樹。無向圖的DFS樹只有Tree Edge和Back Edge,所以一個節點是不可能透過其他子樹回到祖先的,如果不經過父節點的情況下,無法走回祖先,那麼該節點的父節點就是割點了。所以要判斷一個點是不是割點,就要判斷它的子節點能不能不經由它會到祖先,tanjan演算法定義一個$low$函數,$low$函數的定義為一個點透過自己本身和子孫,能走到最小祖先的深度,$low$函數可以在DFS時一並計算。\\
有了$low$函數,就可以判斷子節點是否不透過自己連回祖先。對於一個非根節點,如果有任一子節點不能不透過自己連回祖先,該點就是割點。對於根節點,如果子節點$>1$,該點就是割點。\\
我們整理成以下步驟:
\begin{enumerate}
\item [DFS] 在DFS過程維護$Low$函數
\item [非根節點] 如果有任一子節點不能不透過自己連回祖先,該點就是割點
\item [根節點] 如果子節點$>1$,該點就是割點
\end{enumerate}
如果計算樹邊,那麼兒子的$low$函數只要$>=$父親自己的深度,如果不算樹邊的話,那麼兒子的$low$函數需嚴格大於父親自己的深度。
\lstinputlisting{Graph/UdgCutVertex.cpp}
這個演算法主要是做DFS,所以時間複雜度為$O(V+E)$
\subsubsection{邊雙連通}
和點連通相似,要判斷一張圖是否邊雙連通,就要檢查他是否有割邊,如果沒有割邊,則這張圖為邊雙連通。
\begin{enumerate}
\item [割邊] 給定一張圖$G$,如果移除邊$e$及$e$連接的點之後,圖$G$不再連通,邊$e$都被稱為$G$的一個割邊(cut-edge)或橋(bridge)。
\end{enumerate}
樹邊才有可能是橋,其他的邊拔除仍然可以藉由樹邊連通。類似地如果子節點只能從父親到它這條邊去走回祖先,那麼父親到兒子的這條邊就是橋。我們同樣可以用定義並維護$low$函數,但是根節點不再是特例。
\lstinputlisting{Graph/UdgCutEdge.cpp}
和前面點雙連通相同,時間複雜度為$O(V+E)$
\subsubsection{題目}
\begin{enumerate}
\item UVa 11504
\item UVa 10972
\end{enumerate}
\subsubsection{雙連通元件}
先來定義相關術語
\begin{enumerate}
\item [連通元件] 一張圖 $G$ 有很多子圖,如果一個子圖 $G'$ 是連通的,我們稱之為連通元件(connected component),如果一個連通元件滿足"加上任意一個其他的點就不再連通",則稱這樣的連通元件是"極大的"(maximal)。
\item [邊雙連通元件] 如果一張的某個子圖是一張邊雙連通圖,我們就成這張子圖為邊連通元子圖(bi-edge-connected graph)或邊雙連通元件(bi-edge-connected component)。
\item [點雙連通元件] 如果一張的某個子圖是一張點雙連通圖,我們就成這張子圖為點連通元子圖(bi-vertex-connected graph)或點雙連通元件(bi-vertex-connected component)。
\end{enumerate}
一般來說,我們會討論極大的連通元件,以下所有的連通元件都是極大的。\\
要一張張圖的所有邊雙連通元件,只要拔掉所有橋,剩下的圖就是原圖所有的邊連通元件。\\
至於點連通元件就沒那麼簡單了,因為同一個割點有可能同時存在多的點連通元件內,並且點的相鄰邊不一定不同的點連通元件中,所以我們不能直接拔點拔掉來求出點雙連通元件。\\
至於為什麼會有這樣的差異,是因為一般的圖是以點為主題,邊用來描敘點和點之間的東西。而點連通問題則是以邊為主體,點用來描述邊和邊之間的關係,後者描述的關係不是二元關係,所以讓問題變得複雜。\\
那我們就以邊的角度來思考點雙連通元件,對於一個點$p$,與父親點$f$之間有條邊$e_f$,與兒子點$c$之間有條邊$e_c$,如果$low(c)>depth(p)$(不計算樹邊),則$e_f$和$e_c$不在同一個點連通元件內,反之$e_f$和$e_c$在同一個點雙連通元件內。根據以上性質,我們可以在DFS過程中維護一個stack,紀錄目前經過的邊,當遇到割點時,可以快速找出點雙連通元件。
\lstinputlisting{Graph/UdgBvc.cpp}
這個演算法一樣有做DFS,並且維護一個stack,每條邊都會被丟進去一次,因此時間複雜度為$O(V+E)$。
\subsubsection{題目}
\begin{enumerate}
\item UVa 1108
\item UVa 1464
\end{enumerate}
\subsubsection{有向圖的強連通}
談完了無向圖,就來談論有向圖,有向圖的邊具有方向性,因此比無向圖更難達成"連通"的性質,於是為了跟無向圖做區分,訂了一個術語"強連通"來表示有向圖的連通性。
\begin{enumerate}
\item [強連通] 對於有向圖上的兩點$A,B$,若存在一條路徑從$A$到$B$,以及存在一條路徑從$B$到$A$,則我們稱$A,B$兩點強連通(strongly connected)
\item [強連通圖] 如果一張有向圖上任意兩點皆強連通,則這張圖為強連通圖(strongly connected graph)
\item [強連通元件] 如果一張圖中的某個子圖是一張強連通圖,我們稱這個子圖為強連通子圖(strongly connected subgraph),或是強連通元件(strongly connected component)
\end{enumerate}
強連通為有向圖中很重要的性質,如果將強連通元件各自縮成一點,新圖是一張有向無環圖(Directed Acyclic Graph, DAG)DAG有許多強力性質,可以讓圖上的問題變得有解,有些圖論題目一開始會先找出SCC來解題。
\subsubsection{強連通元件}
Tarjan演算法也可以找出強連通元件,不過邏輯會很複雜,所以我們介紹另一種演算法叫做Kosaraju's algorithm。\\
Kosaraju's algorithm基於觀察到的兩件事而成,第一件事為將原圖每條邊都反向,得到的新圖,所有SCC的位置依舊相同。第二件事為如果我們用"正確的"順序遍歷圖,每次遍歷到的點視為同一個SCC,那麼是有可能可以找出正確的SCC的。\\
我們分成三種情況來討論那樣才是正確的遍歷順序。
\begin{enumerate}
\item $A,B$在同一個SCC裡:先拜訪誰都可以,反正另外一個點也會被拜訪到
\item $A,B$互相都不能走到對方:這種情況也是先拜訪誰都可以,反正毫不相干
\item $A$走的到$B$,$B$走不到$A$(單向通行):這種情況只能先走$B$再走$A$,否則先走$A$的話,$B$會被認為和$A$在同一個$SCC$內
\end{enumerate}
所以只要給出一個順序,對於所有只有$A$走的到$B$,$B$走不到$A$的點對$(A,B)$都滿足$B$會比$B$先被走到,這個演算法就行得通了。\\
強大的Kosaraju就發明了一個方法:
\begin{enumerate}
\item 將圖上所有邊反向,得到新圖$G'$
\item 在圖$G'$上找一個未拜訪過的點DFS並且記錄時間戳,DFS完後,如果還有點未被DFS,再進行前敘動作。
\item 依時間戳的離開戳記對點由大到小排序,所得序列即為所求。再根據這個序列在原圖$G$做DFS,每次DFS到的點形成一個SCC。
\end{enumerate}
現在來證明這個序列滿足我們的要求:
\begin{enumerate}
\item 對於圖G上任意單向通行的點對$A\rightarrow B$,在$G'$上會變成單向通行的點對$B\rightarrow A$
\item 如果$A$先被拜訪,因為$A$沒辦法做到$B$,所以$A$會先拜訪完畢,因此$A$的離開戳記會小於$B$的離開戳記
\item 如果$B$先被拜訪,$B$一定會走到$A$,$A$拜訪完畢時,$B$一定還沒拜訪完畢,因此$A$的離開戳記依然會小於$B$的離開戳記
\item 得證$A$的離開戳記一定會小於$B$的離開戳記,即$B$在序列中會在$A$前面。
\end{enumerate}
以下是程式碼,此算法會做兩次$DFS$,時間複雜度為$O(V+E)$。
\lstinputlisting{Graph/SCC.cpp}
\subsubsection{題目}
\begin{enumerate}
\item UVa 11324
\item UVa 11504
\end{enumerate}