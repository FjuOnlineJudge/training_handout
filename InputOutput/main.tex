\section{標準輸入輸出}
\subsection{標準輸入輸出、文件輸入輸出}
標準輸入:由鍵盤輸入。\\
標準輸出:輸出到螢幕。\\
文件輸入:由文件輸入。\\
文件輸出:輸出到文件。\\
標準/文件輸出輸入可以混用。,
\subsection{C 標準輸入輸出, scanf 和 printf}
scanf 和 printf 定義在標頭檔 stdio.h。
scanf 和 printf 會給定一個有格式的字串格式為:"\%[*][width][length]specifier",在這個格式後面接參數,在格式字串有幾個參數,後面就要接相對應個數的參數。\\
\lstinputlisting{InputOutput/scanfPrintfBrief.cpp}
specifier(格式碼),是用來指定輸入的型態,下表為常用的格式碼:\\
\begin{tabular}{|r|l|} \hline 格式碼 & 說明 \\\hline \%d & int \\\hline \%u & unsigned int \\\hline \%lld & long long int \\\hline \%llu & unsigned long long int \\\hline \%f & float \\\hline \%lf & double \\\hline \end{tabular}\\
\lstinputlisting{InputOutput/scanfPrintf.cpp}
上面的例子會發現 scanf 的變數前面都會多一個 \&,簡單來說就是 scanf 要的是記憶體位置,要更詳細理解要先學完指標。(編者認為這是當初設計沒想到的一個 bug) \\
width 在 printf 的用途為對右靠齊,如果輸出的字元數小於所設寬度,不足部分會用空白填滿;如果輸出的字元數大於所設寬度,則無效。\\
\lstinputlisting{InputOutput/printhWidth.cpp}
若無法事先知道寬度,可用 "*" 代替。\\
\lstinputlisting{InputOutput/printhWidth2.cpp}
length 在 printf 的用途為指定輸出長度。整數部分,如果是輸出的字元數小於所設長度,不足部分會補 $0$;如果輸出的字元數大於所設長度,則無效。浮點數部分,所謂的輸出長度指的是小數點後的位數。\\
\lstinputlisting{InputOutput/printhLength.cpp}
若無法事先知道長度,亦可用 "*" 代替。\\
\lstinputlisting{InputOutput/printhLength2.cpp}
width 在 scanf 的用法,請參考 C 字串章節。\\
\subsection{C++ 輸入輸出, cin 和 cout}
cin 和 cout 定義在標頭檔 iostream。\\
cin 和 cout 並不需要先給出格式,直接接你要輸出/輸入的變數名稱就好了。\\
\lstinputlisting{InputOutput/cinCoutBrief.cpp}
\lstinputlisting{InputOutput/cinCout.cpp}
C++ 的 cout 也有設定寬度和長度的功能,需使用到標頭檔 iomanip。\\
寬度:setw,如果輸出的字元數小於所設寬度,不足部分會用空白填滿;如果輸出的字元數大於所設寬度,則無效\\
\lstinputlisting{InputOutput/setw.cpp}
長度:setprecision,只對浮點數有效。cout 有兩種表示方式,scientific 和 fixed,scientific 只會輸出有效位數,fixed 會將所有位數印出。\\
\lstinputlisting{InputOutput/setprecision.cpp}