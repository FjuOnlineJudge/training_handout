\section{基本}
\subsection{複雜度}
複雜度是定性描述該演算法執行成本(時間/空間)函式,用來分析資料結構和演算法(DSA)。
\subsubsection{常用函數}
\begin{enumerate}
\item [Big $O$]
用來表示一個複雜度的上界,定義為$f(n)\in O(g(n))\ \ iff\ \ \exists\ c,\ N\in R^{+},\ \forall n \geq  N$ 有 $|f(n)| \leq |cg(n)|$,例如$f(n)=5n^2+4n+1$,我們會注重最高項$5n^2$,且我們會5是常數,得出$f(n)\in O(n^2)$
\item [Big $\Omega$]
用來表示一個複雜度的下界,對於任意的 $f(n) \in O(g(n))$,都有 $g(n) \in \Omega (f(n))$。
\item [Big $\Theta$]
要同時滿足Big $O$和Big $\Omega$
\end{enumerate}
Big $O$ 是我們比較常用的,其他兩個可能再一些地方會用到
    
\subsubsection{常見複雜度}
$O(1) < O(log\ n) < O(n) < O(n\ log\ n) < O(n^2) < O(n^3) < O(2^n) < O(n!)$
另外還有一個在並查集常見,即$O(\alpha(n))$,近似於$O(1)$,可直接當作$O(1)$
\subsubsection{時間/空間複雜度}
時間複雜度,和運算有關,*/\%會比+-還要久,而複雜度得項次會跟迴圈有關,初階競賽只會在意你的項次,只要不要太大基本都會過,進階些比賽,有可能出現常數過大,導致複雜度合理卻還是吃TLE的情況,這時候需要利用"壓常數"技巧,降低時間,讓程式AC。 \\  空間複雜度,則是跟你宣告的變數記憶體總和有關,比時間複雜度容易估計,在樹狀的資料結構,往往需要搭配動態記憶體,才不會因為開太多空間而吃了MLE。 \\  題外話,如果你在你的array不是開在全域內,開了10的5,6次,在執行時跑出RE,那你有以下兩種解決方式
\begin{enumerate}
\item 把array移至全域
\item 加上static,表示靜態變數
\end{enumerate}
\lstinputlisting{Syntax/static.cpp}
\subsection{函式}
函式為程式裡的運算單元,可以接受資料,並回傳指定值。main是C/C++程式的入口函式,接受命令列的參數,正常情況會回傳0代表正常運作。\\
以下為其語法
\lstinputlisting{Syntax/function0.cpp}
範例
\lstinputlisting{Syntax/function1.cpp}
函式有個特性為自呼叫,也就是自己的區域可以呼叫自己,但要有終止條件,不然會陷入無限遞迴,同時也要避免遞迴過深,造成stack overflow。
\lstinputlisting{Syntax/function2.cpp}
函式有很多用處,一個為模組化,意即相同的部分(最多只差一些參數),寫成一個函式,除了簡潔,在除錯也比較方便。一個是利用自呼叫特性實作遞迴,遞迴可將問題拆解成同類的子問題而解決問題。\\
常見遞迴使用
\begin{enumerate}
\item 分治
\item dp中的top-down
\item 圖/樹的搜索
\end{enumerate}

\subsection{指標}
指標是紀錄記憶體的位址的變數,不管是基礎型態或自定義型態皆可用指標,指標的可以讓你直接對記憶體操作。而指標對學習者是一到難度高的門檻,但在程式競賽中,是不可或缺的。\\ 指標在程競中會用到的地方是"動態記憶體配置",這在比較進階的資料型態會比較常出現。
\lstinputlisting{Syntax/pointer.cpp}
\lstinputlisting{Syntax/pointer2.cpp}
\subsection{參考}
參考型態代表一個變數的別名,可直接取得變數的位址,並間接透過參考型態別名來操作物件, 作用類似於指標,但卻不必使用指標語法,也就是不必使`*`運算子來提取值。
\lstinputlisting{Syntax/reference.cpp}
參考型態可用在取代太長的變數(如:`a[x][y][z]`),容易維護。另一個是當函式要傳入可修改的值,可取代指標。
\subsection{傳值}
函式傳入的參數,可以是一般、指標或是參考型態,以下以Swap來介紹
\subsubsection{call by value}
傳入的變數為一般型態,會"複製"一份到函式,原本的變數不會有任何改變。
\lstinputlisting{Syntax/pass1.cpp}
\subsubsection{call by address/value of pointer}
傳入的變數為指標型態,函式內的變數改變,是對記憶體操作,所以原本的數字也會跟著改變。
\lstinputlisting{Syntax/pass2.cpp}
\subsubsection{call by reference}
傳入的變數為參考型態,函數內的變數是原本變數的分身,所以函數內變數改變時,原本變數也會跟者改變。
\lstinputlisting{Syntax/pass3.cpp}
 \subsection{struct}
structc是讓coder能將原本獨立的資料包在一起。例如:三維空間由x座標、y座標、z座標組成。\\
語法:
\begin{enumerate}
\item 型態(type)可以是一般或是指標型態
\item 也可以寫函式或重載運算子
\end{enumerate}
\lstinputlisting{Syntax/struct1.cpp}
以下的例子為平面上的點。
\lstinputlisting{Syntax/struct2.cpp}
\subsubsection{建構子(constructor)、解構子(destructor)}
建構子和strcut name同名,是用來初始化struct裡的資料,如果不寫的話,會有預設建構子,裡面的資料都是亂數。根據請況可多載,然而,如果你寫了運算子,一定要寫一個不帶任何參數的運算子,否則的話,像第14行這樣只有宣告,沒加其他東西的的程式碼就不會通過。\\
解構子的名字形式為 $ \sim strcut name$,是在變數離開作用域時運作,不寫的話也是會有預設解構子,在程式比賽中這樣就已足夠。
\subsubsection{重載運算子}
c++原有的型態都根據需要,定義了各種運算子,但structㄒ如果有需要的話,須自己定義。而在競賽中,常需要作排序而需要小於運算子(`<`)。
\subsection{algorithm}
\subsubsection{sort}
這個函式傳入兩個變數,代表容器(array 或是 vector)的頭尾$[a,b)$,這裡的b不會排序,用來指示為結尾,例如要排序a陣列的第0到第5個元素。
\lstinputlisting{Syntax/sort.cpp}
此函數的複雜度圍$O(n\log n)$,n為排序的個數
\subsubsection{min/max}
min和max原本在c定義在math.h內,c++將它移入algorithm中
\lstinputlisting{Syntax/minmax.cpp}
\subsubsection{lower\_bound/upper\_bound}
這兩個函式會在"有序序列"中尋找值,前兩個值放的是容器(array 或是 vector)的頭尾[a,b),第三的是比較的值val。
\lstinputlisting{Syntax/bound1.cpp}
如果要將位置傳換成數字,直接減起始位置就可,下面是一個範例(from cplusplus):
\lstinputlisting{Syntax/bound2.cpp}
\subsubsection{next\_permutation/prev\_permutation}
這兩個函式會幫你的陣列轉為後/前一個字典序,如果沒有後/前一個字典序,這個函式會回傳false,也不會有任何改變,以下為範例(from cplusplus):
\lstinputlisting{Syntax/permutation.cpp}
\subsection{cmath}
\subsubsection{atan/atan2} 
atan/atan2函數是將斜率轉為弧度,如果要在轉為角度就以180度除以PI就好,而atan直接傳入斜率,atan2則是座標,atan2可以處理x=0的狀況,比atan好用。
\lstinputlisting{Syntax/atan.cpp}
\subsubsection{log/log2/log10}
這些都是常用對數函數分別以e,2,10為底
\lstinputlisting{Syntax/log.cpp}
\subsubsection{pow}
此函數會回傳以base為底的exponent次方,若$\geq 10^6$,就會輸出科學記號。
\lstinputlisting{Syntax/pow.cpp}
\subsubsection{sqrt}
此函數會回傳x的根號次方
\lstinputlisting{Syntax/sqrt.cpp}
\subsection{iomanip}
\subsubsection{setw}
這個函式會將傳入的整數設定輸出寬度後輸出。
\lstinputlisting{Syntax/setw.cpp}
\subsubsection{setprecision}
這個函式設定輸出到小數點後幾位。
\lstinputlisting{Syntax/setprecision.cpp}
\subsection{extra syntax}
\subsubsection{break, continue, return}
\begin{enumerate}
\item break:跳出迴圈
\item continue:這輪不做,到下一輪
\item return:跳出函式,並回傳值
\end{enumerate}
在break, continue, return後的else是無用的,因為如果這三種指令執行了,後面的東西就根本不會執行到。
\subsubsection{const}
const用途在於宣告這個變數式不能更動的,這適合用來宣告常數。
\lstinputlisting{Syntax/const.cpp}
\subsubsection{static}
如果一個變數被宣告為static,那麼他只會被宣告一次,直到整個程式結束才被刪除。
\lstinputlisting{Syntax/static2.cpp}
\subsubsection{define}
不知道各位有沒有用過excel中的巨集,他可以幫你做重複性高的動作,C++中的define可以幫你做類似的事。
\lstinputlisting{Syntax/define.cpp}
由上面範例可見,define 可以取代程式中出現的特定字元,還可以帶參數,為了要讓使用define後的結果是正確的,請將取代後的字元括號起來,否則會輸出非預期的結果如上面範例第9行
\lstinputlisting{Syntax/define.cpp}
\subsubsection{typedef}
typedef 可以為型態取別名,在之後用到要宣告該型態的時候,可以打該型態之別名,減省時間。C++11開始可以用using來達到相同的事。
\lstinputlisting{Syntax/typedef.cpp}
\subsubsection{auto}
C++11開始,新增了一個關鍵字叫auto,auto可以自動判別變數型態,但必須給他初始值,否則他無法判別型態,C++14開始,可用在function回傳值
\lstinputlisting{Syntax/auto.cpp}
\subsubsection{range\_based for}
C++11開始有另外一種for是range\_based,他只要給兩個參數,一個變數指定資料型態並提供遍歷,另一個為要遍歷的範圍,例子如下。
\lstinputlisting{Syntax/rangebasefor.cpp}
\subsection{lambda}
方便地定義匿名函式
\subsubsection{lambda-introducer}
也叫Capture clause,宣告外部變數(在可視範圍(scope)內)傳入此函式內的方法。
\begin{enumerate}
\item `[]`:只有兩個中括號,完全不抓取外部的變數。
\item `[=]`:所有的變數都以傳值(call by value)的方式抓取
\item `[\&]`:所有的變數都以傳參考(call by reference)的方式抓取
\item `[x,\&y]`:x變數使用傳值,y變數使用傳參考
\item `[=,\&y]`:除了y變數使用傳參考以外。其餘的變數皆使用傳值的方式
\item `[\&,x]`:除了x變數使用傳值以外,其餘的變數皆使用傳參考的方式
\end{enumerate}
\subsubsection{lambda declarator}
也叫參數清單,傳入此函式對應資料。
\subsubsection{mutable specification}
指定以傳值方式抓取進來的外部變數,如果用不到可省略。
與一般函數的傳入參數之異
\begin{enumerate}
\item 不可指定參數的預設值。
\item 不可使用可變長度的參數列表。
\item 參數列表不可以包含沒有命名的參數。
\end{enumerate}
\subsubsection{例外狀況規格}
指定該函示會丟出的例外,其使用的方法跟一般函數的例外指定方式一樣,如果用不到可省略。
\subsubsection{傳回值型別}
指定lambda expression傳回型別,如果 lambda expression 所定義的函數很單純,只有包含一個傳回陳述式(statement)或是根本沒有傳回值的話,可省略(optional)
\subsubsection{compound-statement}
亦稱為 Lambda 主體(lambda body),跟一般的函數內容一樣。\\ \\
最後來看個lambda範例,結束這一章節。
\lstinputlisting{Syntax/lambda1.cpp}
\lstinputlisting{Syntax/lambda2.cpp}