\section{其他}
\section{演算法}
\subsection{何謂演算法}
簡言之就是解決問題的方法,用程式語言把他明確地列出。
\subsection{枚舉}
枚舉是最直觀的演算法,將有可能的答案都搜過一遍,當然沒有頭緒的搜尋可能會得到龐大的複雜度,要根據題目的性質來降低複雜度。
\subsubsection{回朔}
枚舉有時能用遞迴實作,在遇到不可能的情形馬上回傳,這種方法就叫做回朔ㄊ。
\subsubsection{特殊枚舉方式}
\begin{enumerate}
\item [二進位] 利用二進位來表示集合內有哪些元素要用,進而枚舉所有元素子集,但受限於時間複雜度$O(2^n)$,集合的元素個數通常只有30個(甚至15個)。   
\item [字典序] 利用 next\_permutation 或 prev\_permutation 達到枚舉元素的先後順序。時間複雜度為$O(N!)$
\end{enumerate}
% \subsubsection{爬行法}
\subsubsection{折半枚舉}
有時遇到複雜度$O(2^n)$的算法,在無法用其他方法降低複雜度,可以試著將元素切成兩半,降低n,再用其他算法組合起來。
\subsubsection{題目}
\begin{enumerate}
\item UVa 11059(區間列舉)
\item UVa 1481(區間列舉)
\item UVa 10976(減少列舉範圍)
\item UVa 750(回朔)
\item UVa 524(回朔)
\item UVa 11464
\item UVa 1326(折半枚舉)
\end{enumerate}
\subsection{貪心}
對於一個問題,始終使用同一種方法,採取在目前狀態下最好或最佳(即最有利)的選擇。\\
有的貪心很直觀,有的就需要通靈才解得出來,往往做題目一開始想到的辦法是錯的,直到做到一半才發現。所以我們需要證明方法是不是對的,這往往需要時間練習,才不會到比賽遇到時,花了很多時間去解題。\\
\subsubsection{證明的辦法}
\begin{enumerate}
\item 試圖構造出反例,發現他不存在。
\item 如果存在更佳解的答案比你做出來的還好,那這組解一定可以再做得更好,進而達到反證出更佳解不存在。
\item 使用遞迴證法:(1) 證明基底是對的。(2) 假設小問題是好的。(3) 你一定可以用最好的方法來將問題簡化成剛才假設是好的小問題。
\end{enumerate}
\subsubsection{題目}
\begin{enumerate}
\item UVa 11729
\item UVa 11292
\item UVa 11389
\item UVa 1445
\item UVa 993
\item TIOJ 1441
\end{enumerate}
\subsection{二分搜}
對於一個函數$F(n)$,如果存在一個x,對於所有 $\geq x$ 的a,$F(a)=$ true,反之$F(a)=$ false,基於這樣的單調性,就可以用二分搜。
\lstinputlisting{BasicA/binarySearch.cpp}
有些題目為"最多/最少為何會成立",那麼如果你可以在良好的時間檢查出"如果代價是 x,那可不可以達成目標",並且x具有單調性,那麼你可以轉換成"如果代價是 x,那可不可以達成目標"傳換成$F(x)$,對答案(x)進行二分搜。\\
二分搜要注意兩件事,一個是無限迴圈,要避免它可以在腦中先模擬一下。一個是在實數中二分搜,因為實數的稠密性,題目會有誤差容忍(例如$10^{-6}$),只要在誤差內都是容許的。
\subsubsection{三分搜}
對於U型函數(例如$y=F(x)=x^2$),我們想要找尋其極值,意謂其左右兩側皆各自遞增/遞減,我們可以利用三分搜來解決(二分搜只能解決全體單調性,不能解決有兩邊的)。\\
考慮三分後從左到右四個採樣點的關係
\begin{enumerate}
\item $S(a) < S(b) < S(c) < S(d)$,此時最小值一定不在最右邊
\item $S(a) > S(b) < S(c) < S(d)$,此時最小值一定不在最右邊
\item $S(a) > S(b) > S(c) < S(d)$,此時最小值一定不在最左邊
\item $S(a) > S(b) > S(c) > S(d)$,此時最小值一定不在最左邊
\end{enumerate}
這段描敘還可以再簡化
\begin{enumerate}
\item $S(b) < S(c)$,此時最小值一定不在最右邊
\item $S(b) > S(c)$,此時最小值一定不在最左邊
\end{enumerate}
每次都至少可以讓區間縮小$\frac{1}{3}$
\lstinputlisting{BasicA/3Search.cpp}
\subsubsection{題目}
\begin{enumerate}
\item Uva 714
\item Uva 1421
\item Uva 11627
\item zerojudge d732
\item neoj 72(三分搜)
\end{enumerate}
\subsection{分治}
分治法會把問題分解成子問題(分),解決完再合併回原本的問題(治)。\\
分治分成以下步驟
\begin{enumerate}
\item 切割:把一個問題切成子問題然後遞迴
\item 碰底:碰到不能再切割或是明顯有答案(也許無解),就算出答案再回傳
\item 合併:利用傳回來的子問題算出答案然後回傳
\end{enumerate}
\subsubsection{合併排序法}
一個利用分治實作的排序法,逆序數對也會利用他的概念來實作。
\begin{enumerate}
\item 切割:把序列分成兩半然後遞迴
\item 碰底:直到序列長度為1,這時候已為一個排好的序列,直接回傳
\item 合併:利用傳回來的兩串序列進行排序
\end{enumerate}
\lstinputlisting{BasicA/mergesort.cpp}
\subsubsection{更多的經典題目}
\begin{enumerate}
\item 快速排序法。
\item 逆序數對。(經典問題,搭配 Merge Sort)
\end{enumerate}
\subsubsection{題目}
\begin{enumerate}
\item uva 1608
\item uva 10810(逆序數對)
\item uva 11129
\item uva 10245
\end{enumerate}

指標是紀錄記憶體的位址的變數,不管是基礎型態或自定義型態皆可用指標,指標的可以讓你直接對記憶體操作。而指標對學習者是一到難度高的門檻,但在程式競賽中,是不可或缺的。\\ 指標在程競中會用到的地方是"動態記憶體配置",這在比較進階的資料型態會比較常出現。
\lstinputlisting{Syntax/pointer.cpp}
\lstinputlisting{Syntax/pointer2.cpp}
\subsection{algorithm}
\subsubsection{sort}
這個函式傳入兩個變數,代表容器(array 或是 vector)的頭尾$[a,b)$,這裡的b不會排序,用來指示為結尾,例如要排序a陣列的第0到第5個元素。
\lstinputlisting{Syntax/sort.cpp}
此函數的複雜度圍$O(n\log n)$,n為排序的個數
\subsubsection{min/max}
min和max原本在c定義在math.h內,c++將它移入algorithm中
\lstinputlisting{Syntax/minmax.cpp}
\subsubsection{lower\_bound/upper\_bound}
這兩個函式會在"有序序列"中尋找值,前兩個值放的是容器(array 或是 vector)的頭尾[a,b),第三的是比較的值val。
\lstinputlisting{Syntax/bound1.cpp}
如果要將位置傳換成數字,直接減起始位置就可,下面是一個範例(from cplusplus):
\lstinputlisting{Syntax/bound2.cpp}
\subsubsection{next\_permutation/prev\_permutation}
這兩個函式會幫你的陣列轉為後/前一個字典序,如果沒有後/前一個字典序,這個函式會回傳false,也不會有任何改變,以下為範例(from cplusplus):
\lstinputlisting{Syntax/permutation.cpp}
\subsection{cmath}
\subsubsection{atan/atan2} 
atan/atan2函數是將斜率轉為弧度,如果要在轉為角度就以180度除以PI就好,而atan直接傳入斜率,atan2則是座標,atan2可以處理x=0的狀況,比atan好用。
\lstinputlisting{Syntax/atan.cpp}
\subsubsection{log/log2/log10}
這些都是常用對數函數分別以e,2,10為底
\lstinputlisting{Syntax/log.cpp}
\subsubsection{pow}
此函數會回傳以base為底的exponent次方,若$\geq 10^6$,就會輸出科學記號。
\lstinputlisting{Syntax/pow.cpp}
\subsubsection{sqrt}
此函數會回傳x的根號次方
\lstinputlisting{Syntax/sqrt.cpp}
\subsection{iomanip}
\subsubsection{setw}
這個函式會將傳入的整數設定輸出寬度後輸出。
\lstinputlisting{Syntax/setw.cpp}
\subsubsection{setprecision}
這個函式設定輸出到小數點後幾位。
\lstinputlisting{Syntax/setprecision.cpp}
\subsection{extra syntax}
\subsubsection{break, continue, return}
\begin{enumerate}
\item break:跳出迴圈
\item continue:這輪不做,到下一輪
\item return:跳出函式,並回傳值
\end{enumerate}
在break, continue, return後的else是無用的,因為如果這三種指令執行了,後面的東西就根本不會執行到。
\subsubsection{const}
const用途在於宣告這個變數式不能更動的,這適合用來宣告常數。
\lstinputlisting{Syntax/const.cpp}
\subsubsection{static}
如果一個變數被宣告為static,那麼他只會被宣告一次,直到整個程式結束才被刪除。
\lstinputlisting{Syntax/static2.cpp}
\subsubsection{define}
不知道各位有沒有用過excel中的巨集,他可以幫你做重複性高的動作,C++中的define可以幫你做類似的事。
\lstinputlisting{Syntax/define.cpp}
由上面範例可見,define 可以取代程式中出現的特定字元,還可以帶參數,為了要讓使用define後的結果是正確的,請將取代後的字元括號起來,否則會輸出非預期的結果如上面範例第9行
\lstinputlisting{Syntax/define.cpp}
\subsubsection{typedef}
typedef 可以為型態取別名,在之後用到要宣告該型態的時候,可以打該型態之別名,減省時間。C++11開始可以用using來達到相同的事。
\lstinputlisting{Syntax/typedef.cpp}
\subsubsection{auto}
C++11開始,新增了一個關鍵字叫auto,auto可以自動判別變數型態,但必須給他初始值,否則他無法判別型態,C++14開始,可用在function回傳值
\lstinputlisting{Syntax/auto.cpp}
\subsubsection{range\_based for}
C++11開始有另外一種for是range\_based,他只要給兩個參數,一個變數指定資料型態並提供遍歷,另一個為要遍歷的範圍,例子如下。
\lstinputlisting{Syntax/rangebasefor.cpp}
\subsection{lambda}
方便地定義匿名函式
\subsubsection{lambda-introducer}
也叫Capture clause,宣告外部變數(在可視範圍(scope)內)傳入此函式內的方法。
\begin{enumerate}
\item `[]`:只有兩個中括號,完全不抓取外部的變數。
\item `[=]`:所有的變數都以傳值(call by value)的方式抓取
\item `[\&]`:所有的變數都以傳參考(call by reference)的方式抓取
\item `[x,\&y]`:x變數使用傳值,y變數使用傳參考
\item `[=,\&y]`:除了y變數使用傳參考以外。其餘的變數皆使用傳值的方式
\item `[\&,x]`:除了x變數使用傳值以外,其餘的變數皆使用傳參考的方式
\end{enumerate}
\subsubsection{lambda declarator}
也叫參數清單,傳入此函式對應資料。
\subsubsection{mutable specification}
指定以傳值方式抓取進來的外部變數,如果用不到可省略。
與一般函數的傳入參數之異
\begin{enumerate}
\item 不可指定參數的預設值。
\item 不可使用可變長度的參數列表。
\item 參數列表不可以包含沒有命名的參數。
\end{enumerate}
\subsubsection{例外狀況規格}
指定該函示會丟出的例外,其使用的方法跟一般函數的例外指定方式一樣,如果用不到可省略。
\subsubsection{傳回值型別}
指定lambda expression傳回型別,如果 lambda expression 所定義的函數很單純,只有包含一個傳回陳述式(statement)或是根本沒有傳回值的話,可省略(optional)
\subsubsection{compound-statement}
亦稱為 Lambda 主體(lambda body),跟一般的函數內容一樣。\\ \\
最後來看個lambda範例,結束這一章節。
\lstinputlisting{Syntax/lambda1.cpp}
\lstinputlisting{Syntax/lambda2.cpp}