\subsection{複雜度}
複雜度是定性描述該演算法執行成本(時間/空間)函式,用來分析資料結構和演算法(DSA)。
\subsubsection{常用函數}
\begin{enumerate}
\item [Big $O$]
用來表示一個複雜度的上界,定義為$f(n)\in O(g(n))\ \ iff\ \ \exists\ c,\ N\in R^{+},\ \forall n \geq  N$ 有 $|f(n)| \leq |cg(n)|$,例如$f(n)=5n^2+4n+1$,我們會注重最高項$5n^2$,且我們會5是常數,得出$f(n)\in O(n^2)$
\item [Big $\Omega$]
用來表示一個複雜度的下界,對於任意的 $f(n) \in O(g(n))$,都有 $g(n) \in \Omega (f(n))$。
\item [Big $\Theta$]
要同時滿足Big $O$和Big $\Omega$
\end{enumerate}
Big $O$ 是我們比較常用的,其他兩個可能再一些地方會用到
    
\subsubsection{常見複雜度}
$O(1) < O(log\ n) < O(n) < O(n\ log\ n) < O(n^2) < O(n^3) < O(2^n) < O(n!)$
另外還有一個在並查集常見,即$O(\alpha(n))$,近似於$O(1)$,可直接當作$O(1)$
\subsubsection{時間/空間複雜度}
時間複雜度,和運算有關,*/\%會比+-還要久,而複雜度得項次會跟迴圈有關,初階競賽只會在意你的項次,只要不要太大基本都會過,進階些比賽,有可能出現常數過大,導致複雜度合理卻還是吃TLE的情況,這時候需要利用"壓常數"技巧,降低時間,讓程式AC。 \\  空間複雜度,則是跟你宣告的變數記憶體總和有關,比時間複雜度容易估計,在樹狀的資料結構,往往需要搭配動態記憶體,才不會因為開太多空間而吃了MLE。 \\  題外話,如果你在你的array不是開在全域內,開了10的5,6次,在執行時跑出RE,那你有以下兩種解決方式
\begin{enumerate}
\item 把array移至全域
\item 加上static,表示靜態變數
\end{enumerate}
\lstinputlisting{Syntax/static.cpp}